\documentclass[openany]{article}
\usepackage[a4paper,margin=1in,bottom=1.5in]{geometry} % define margins. Bottom margin is used to lift a little bit the page number.
\usepackage[english]{babel} % document language is english
\usepackage{tikz} % for drawing (currently not used).
\usepackage{graphicx} % for including images
\usepackage[export]{adjustbox}
\usepackage{fancyhdr} % used for creating headers and footers. only used in title page in this document.
\usepackage{tabularx} % creation of more complex tables
\usepackage{longtable} % tables can span multiple pages
\usepackage{array} % allow elements of tabular environment to have vertical alignment, e.g., center alignment.
\usepackage{nameref} % make it possible to reference by name
\usepackage{hyperref} % allow hiperlinks (links to other document parts and extern links)
\usepackage{etoc} % used for generation of section local table of contents
\usepackage{placeins}
\usepackage{siunitx} % SI units package
\usepackage{enumitem} % allows removing space between list items
\usepackage{xcolor,colortbl} % makes it possible to change table lines color

% Define graphics path
\graphicspath{{figs/}}

% Configure the cross reference hyper links color
\hypersetup{
    colorlinks=true,
    linkcolor=blue,
}

\renewcommand{\arraystretch}{2} % increase height of table rows
\newcolumntype{N}{p{14cm}} % new column type

\newcolumntype{C}{>{\centering\arraybackslash}X} % new column type for tabularx
						 % centered (\centering), adjust width in order to fill table width (X type)

% Configure header in 'titlepage'
%\pagestyle{fancy}
%\lhead{\includegraphics[width=4.5cm]{logo_cnpem}}
%\rhead{\includegraphics[width=4cm]{logo_lnls}}
%\renewcommand{\headrulewidth}{0pt}
%\setlength{\headheight}{52pt}
% Clean footer
%\fancyfoot{}

% increase table height factor a little bit (taller cells)
%\renewcommand{\arraystretch}{1.5}

%==== Begin DOCUMENT ====
\begin{document}

%--- Begin title page ---
\begin{titlepage}

% Add header to this page
%\thispagestyle{fancy}

% Center elements
\begin{center}

% title of title page
\topskip0pt % perfectly centered
\vspace*{\fill}
\textbf{\Huge RSSMX100A EPICS IOC User Guide}\\[20pt]
\textbf{\Huge Version 1.0}\\[20pt]
\textbf{\Huge January/2018}
\vspace*{\fill}

% footer of title page
\vfill
\textbf{Beam Diagnostics Group (DIG)}\\[5pt]
\textbf{Brazilian Synchrotron Light Laboratory (LNLS)}\\[5pt]
\textbf{Brazilian Center for Research in Energy and Materials (CNPEM)}
\end{center}

\end{titlepage}
%--- End of title page ---

\newpage
\pagestyle{plain} % restore default page style

%--- Table of contents ---
\tableofcontents

\newpage
%--- Section: RSSMX100A IOC ---
\section{RSSMX IOC}

	\paragraph{} The RSSMX100A IOC provides most of the SMA and SMB signal generator parameters as EPICS PVs. Its goal is to facilitate the process of setting the desired signal parameters and building application-specific IOCs which could make use of a SMX general IOC.

%--- Section: Document Overview ---
\section{Document Overview}

	\paragraph{} This document lists the IOC PVs along with their data type, limits, units, description, and related command. In most cases, a PV is a direct mapping of a SMA or SMB parameter, and its description is the same provided for the parameter in the Manufacturer Reference Manual. The Reference Manual provides all the information about the signal generator features and options. After a function or parameter is well understood, it should be easy to locate the associated PVs on this document.

%--- Section: IOC Configuration Steps ---
\section{IOC Configuration Steps}

	% Dependencies
	\paragraph{} This IOC requires \emph{EPICS base 3.14.12.5} and \emph{synApps 5.8}.

	% Edit Release File
	\paragraph{} When setting up the IOC, it is necessary to edit the \emph{RELEASE} file in the \emph{configure} directory to provide the right path to support modules. Edit the following lines:

	\begin{itemize}
		\item[] SUPPORT=/\textless path\textgreater/\textless to\textgreater/\textless synApps\textgreater
		\item[] EPICS\_BASE=/\textless path\textgreater/\textless to\textgreater/\textless epics\textgreater/\textless base\textgreater
		\item[] ASYN=\$(SUPPORT)/\textless path to asyn\textgreater
		\item[] STREAM=\$(SUPPORT)/\textless path to stream device\textgreater
		\item[] CALC=\$(SUPPORT)/\textless path to calc module\textgreater
		\item[] AUTOSAVE=\$(SUPPORT)/\textless path to autosave\textgreater
	\end{itemize}

	% Edit st.cmd file
	\paragraph{} Edit the \emph{st.cmd} file to set the SMX network address using the \emph{drvAsynIPPortConfigure} command. Load the \emph{SMA.db or SMB.db} with the \emph{dbLoadRecords} command and set the desired prefix for the records names. The records' names prefixes have two parts: \emph{P} and \emph{R}.

	\begin{itemize}
		\item[] drvAsynIPPortConfigure("\textless port name\textgreater", "\textless RSSMX100A IP{\textgreater} TCP",0,0,0)
		\item[] dbLoadRecords("\${TOP}/db/SMA.db", "P=\textless first prefix part\textgreater, R=\textless second prefix part\textgreater, PORT=\textless port name\textgreater")
		\item[] dbLoadRecords("\${TOP}/db/SMB.db", "P=\textless first prefix part\textgreater, R=\textless second prefix part\textgreater, PORT=\textless port name\textgreater")
	\end{itemize}

%--- Section: PVs Suffixes ---
\section{PVs Prefixes}

	\paragraph{} The records in this IOC fall into different categories depending on their functionalities. The categories are defined by the prefixes, according to Table 1.

	\begin{table}[!h]
		\center
		\caption{PVs Prefixes}
		\begin{tabular}{m{3cm} m{3cm} m{7cm}}
			\hline
			\bfseries Mnemonic & \bfseries Name & \bfseries Description \\ \hline
			Gen & General & General functionality \\ \hline
			Freq & Frequency & Frequency functionalities \\ \hline
			Mod & Modulation & Modulation functionalities \\ \hline
			Trig & Trigger & Triggering functionalities \\ \hline
			Rosc & Reference Oscillator & Reference Oscillator functionalities \\ \hline
			Csyn & Clock Synthesis & Clock Synthesis functionalities\\ \hline
			Nois & Noise  & Noise functionalities \\ \hline

		\end{tabular}
	\end{table}

%--- Section: PV List ---
\section{PV List}

		\paragraph{} Each PV information block starts with the PV Channel Access name in bold text, followed by a longer, more descriptive name. The PV input data type comes next, followed by the limits or options, when relevant. A description of the PV function is then provided. When available, the associated multimeter command (TSP command) is listed. For PVs that only work when certain measure functions are selected, a table indicates the set of allowed functions at the end of the block.

		\newcommand{\FuncTableBorderColor}{gray!50} % define function table border color
		\newcommand{\nofunc}{\cellcolor{gray!20}\color{gray}} % define function table "not allowed function" cell color
		\newcommand{\yesfunc}{\cellcolor{white}\color{black}} % define function table "allowed function" cell color

		\bigskip
		\begin{tabular}{N}
			\hline
			\bfseries PVExample \\ \hline
			\emph{PV Name Example} \\
			Data type: The PV data type. \\
			Unit: show the datas unit, when applicable. \\
			Range: minimun and maximun permitted values, when applicable. \\
			Description: Description of the PV function. \\
			Command: Shows the related equipment remote command. \\
			Configuration: what configuration the equipment must have for this PV to be editable. \\

		\end{tabular}


	% TABLE: General Functionalities
	\subsection{General Functionalities}\label{pvgroup:function}
		

		\paragraph{} % This paragraph aligns the first tabular with the others


		\begin{tabular}{N}
			\hline
			\bfseries GeneralSearchOnce-Cmd \\ \hline
			\emph{Activate Level Control Correction Command} \\
			Data type: bool \{\begin{itemize}[noitemsep]
				\small
				\item[] OFF
				\item[] ON
			\end{itemize}\} \\
			Description: Temporarily activates level control for correction purposes. \\
			Command: SOUR:POW:ALC:SONC \emph{value} \\
			\\

		\end{tabular}


		\begin{tabular}{N}
			\hline
			\bfseries GeneralAttFixLow-Mon \\ \hline
			\emph{Minumum Level With Fixed Attenuator Monitor} \\
			Data type: float \\
			Unit: dB \\
			Description: Queries the minimum level which can be set when the attenuator is fixed. \\
			Command: OUTP:AFIX:RANG:LOW? \\
			Configuration needed: \begin{itemize}[noitemsep]
				\small
				\item[] OUTP:AMOD FIX
			\end{itemize} \\

		\end{tabular}


		\begin{tabular}{N}
			\hline
			\bfseries GeneralFSweep-Sel \\ \hline
			\emph{Frequency Sweep Selection} \\
			Data type: enum \{\begin{itemize}[noitemsep]
				\small
				\item[] CW (OFF)
				\item[] SWEep (ON)
			\end{itemize}\} \\
			Description: Selects frequency sweep mode for the generating RF output signal. The selected mode determines the parameters to be used for further frequency settings. \\
			Command: SOUR:FREQ:MODE \emph{value} \\
			\\

		\end{tabular}


		\begin{tabular}{N}
			\hline
			\bfseries GeneralFSweep-Sts \\ \hline
			\emph{Frequency Sweep Status} \\
			Data type: enum \{\begin{itemize}[noitemsep]
				\small
				\item[] CW
				\item[] SWE
			\end{itemize}\} \\
			Description: Shows the frequency sweep state. \\
			Command: SOUR:FREQ:MODE? \\
			\\

		\end{tabular}


		\begin{tabular}{N}
			\hline
			\bfseries GeneralPwrMan-SP \\ \hline
			\emph{Sweep Step Level Set Point} \\
			Data type: float \\
			Unit: dBm \\
			Range: -145 dBm to 30 dBm \\
			Description: In Sweep mode, this PV sets the level for the next sweep step in the Step sweep mode. Here only level values between the settings of  Start Level (SOUR:POW:STAR) and Stop Level (SOUR:POW:STOP) are permitted. Each sweep step is triggered by a separate command. \\
			Command: SOUR:POW:MAN \emph{value} \\
			Configuration needed:
				\begin{itemize}[noitemsep]
				\small
				\item[] SOUR:POW:MODE SWE
				\item[] SOUR:SWE:POW:MODE MAN
				\end{itemize} \\

		\end{tabular}


		\begin{tabular}{N}
			\hline
			\bfseries GeneralPwrMan-RB \\ \hline
			\emph{Sweep Step Level Read Back} \\
			Data type: float \\
			Unit: dBm \\
			Description: Read the level for the next sweep step in the Step sweep mode. \\
			Command: SOUR:POW:MAN? \\
			\\

		\end{tabular}


		\begin{tabular}{N}
			\hline
			\bfseries GeneralLvlStop-SP \\ \hline
			\emph{Stop Level Set Point} \\
			Data type: float \\
			Unit: dBm \\
			Range: -145 dBm to 30 dBm \\
			Description: Sets the stop level for the RF sweep. It is possible to select any level within the setting range. The range is defined by the Start Level (SOUR:POW:STAR) value and this Stop Level value. A defined offset (SOUR:POW:LEV:IMM:OFFS) affects the level values. \\
			Command: SOUR:POW:STOP \emph{value} \\
			\\

		\end{tabular}


		\begin{tabular}{N}
			\hline
			\bfseries GeneralLvlStop-RB \\ \hline
			\emph{Stop Level Read Back} \\
			Data type: float \\
			Unit: dBm \\
			Description: Read the stop level for the RF sweep. \\
			Command: SOUR:POW:STOP? \\
			\\
			
		\end{tabular}


		\begin{tabular}{N}
			\hline
			\bfseries GeneralRF-Sel \\ \hline
			\emph{RF Enable Selection} \\
			Data type: bool \{\begin{itemize}[noitemsep]
				\small
				\item[] OFF
				\item[] ON
			\end{itemize}\} \\
			Description: Activates and deactivates the RF output signal. \\
			Command: OUTP \emph{value} \\
			\\

		\end{tabular}


		\begin{tabular}{N}
			\hline
			\bfseries GeneralRF-Sts \\ \hline
			\emph{RF Enable Status} \\
			Data type: bool \{\begin{itemize}[noitemsep]
				\small
				\item[] OFF
				\item[] ON
			\end{itemize}\} \\
			Description: shows RF state. \\
			Command: OUTP? \\
			\\

		\end{tabular}

		\begin{tabular}{N}
			\hline
			\bfseries GeneralUsrCorrect-Sel \\ \hline
			\emph{User Correction Enable Selection} \\
			Data type: bool \{\begin{itemize}[noitemsep]
				\small
				\item[] OFF
				\item[] ON
			\end{itemize}\} \\
			Description: Activates/deactivates level correction. \\
			Command: SOUR:CORR:STAT \emph{value} \\
			\\ 

		\end{tabular}


		\begin{tabular}{N}
			\hline
			\bfseries GeneralUsrCorrect-Sts \\ \hline
			\emph{User Correction Enable State} \\
			Data type: bool \{\begin{itemize}[noitemsep]
				\small
				\item[] OFF
				\item[] ON
			\end{itemize}\} \\
			Description: Shows the level correction state. \\
			Command: SOUR:CORR:STAT? \\
			\\

		\end{tabular}


		\begin{tabular}{N}
			\hline
			\bfseries GeneralAlc-Sel \\ \hline
			\emph{Automatic Level Control Selection} \\
			Data type: enum \\
			Description: Activates/deactivates automatic level control. \begin{itemize}[noitemsep]
				\small
				\item[] \textbf{ON}
				\item[] Internal level control is permanently activated.
				\item[] \textbf{OFF}
				\item[] Internal level control is deactivated; Sample and Hold mode is activated.
				\item[] \textbf{AUTO}
				\item[] Internal level control is activated/deactivated automatically, depending on the operatind state.
			\end{itemize} \\
			Command: SOUR:POW:ALC:STAT \emph{value} \\
			\\ 

		\end{tabular}


		\begin{tabular}{N}
			\hline
			\bfseries GeneralAlc-Sts \\ \hline
			\emph{Automatic Level Control Status} \\
			Data type: enum \{\begin{itemize}[noitemsep]
				\small
				\item[] ON
				\item[] OFF
				\item[] AUTO
			\end{itemize}\} \\
			Description: Shows the automatic level control state. \\
			Command: SOUR:POW:ALC:STAT? \\
			\\ 

		\end{tabular}


		\begin{tabular}{N}
			\hline
			\bfseries GeneralDspUpdt-Sel \\ \hline
			\emph{Display Update Enable Selection} \\
			Data type: bool \{\begin{itemize}[noitemsep]
				\small
				\item[] OFF
				\item[] ON
			\end{itemize}\} \\
			Description: Switches the update of the display on/off. A switchover from remote control to manual control always sets the status of the update of the display to ON. \\
			Command: SYST:DISP:UPD \emph{value} \\
			\\ 

		\end{tabular}


		\begin{tabular}{N}
			\hline
			\bfseries GeneralDspUpdt-Sts \\ \hline
			\emph{Display Update Enable Status} \\
			Data type: bool \{\begin{itemize}[noitemsep]
				\small
				\item[] OFF
				\item[] ON
			\end{itemize}\} \\
			Description: Shows the state of the display update. \\
			Command: SYST:DISP:UPD? \\
			\\ 

		\end{tabular}


		\begin{tabular}{N}
			\hline
			\bfseries GeneralFreq-SP \\ \hline
			\emph{Frequency for the RF Signal Set Point} \\
			Data type: float \\
			Unit: Hz \\
			Range: 9 kHz to 3 GHz \\
			Description: Sets the frequency of the RF output signal. In CW mode (FREQ:MODE CW), the instrument operates at a fixed frequency. In sweep mode (FREQ:MODE SWE), the value applies to the sweep frequency and the instrument processes the frequency settings in defined sweep steps.\\
			Command: FREQ \emph{value} \\
			\\
			
		\end{tabular}


		\begin{tabular}{N}
			\hline
			\bfseries GeneralFreq-RB \\ \hline
			\emph{Frequency for the RF Signal Read Back} \\
			Data type: float \\
			Unit: Hz \\
			Description: Reads the frequency of the RF output signal. \\
			Command: FREQ? \\
			\\

		\end{tabular}


		\begin{tabular}{N}
			\hline
			\bfseries GeneralRFLvl-SP \\ \hline
			\emph{Level for the RF Signal Set Point} \\
			Data type: float \\
			Unit: dBm \\
			Range: -145 dBm to 30 dBm \\
			Description: Sets the RF level applied to the DUT (Device Under Test). If specified, a level offset is included. \\
			Command: SOUR:POW:LEV:IMM:AMPL \emph{value} \\
			\\
			
		\end{tabular}


		\begin{tabular}{N}
			\hline
			\bfseries GeneralRFLvl-RB \\ \hline
			\emph{Level for the RF Signal Read Back} \\
			Data type: float \\
			Unit: dBm \\
			Description: Reads the RF level applied to the DUT (Device Under Test). \\
			Command: SOUR:POW:LEV:IMM:AMPL? \\
			\\ 

		\end{tabular}


		\begin{tabular}{N}
			\hline
			\bfseries GeneralPwrLim-SP \\ \hline
			\emph{Limit of Maximum RF Output Level Set Point} \\
			Data type: float \\
			Unit: dBm \\
			Range: -145 dBm to 30 dBm \\
			Description: Limits the maximum RF output level in CW and SWEEP mode. It does not influence the "Level" display or the response to the POW? query command. \\
			Command: SOUR:POW:LIM:AMPL \emph{value} \\
			\\
			
		\end{tabular}


		\begin{tabular}{N}
			\hline
			\bfseries GeneralPwrLim-RB \\ \hline
			\emph{Limit of Maximum RF Output Level Read Back} \\
			Data type: float \\
			Unit: dBm \\
			Description: Reads the limit of maximum RF output level in CW and SWEEP mode. \\
			Command: SOUR:POW:LIM:AMPL? \\
			\\

		\end{tabular}


		\begin{tabular}{N}
			\hline
			\bfseries GeneralSearchOnce-Cmd \\ \hline
			\emph{Activate Level Control Correction Command} \\
			Data type: bool \{\begin{itemize}[noitemsep]
				\small
				\item[] OFF
				\item[] ON
			\end{itemize}\} \\
			Description: Temporarily activates level control for correction purposes. \\
			Command: SOUR:POW:ALC:SONC \emph{value} \\
			\\

		\end{tabular}


	 TABLE: Frequency Functionalities
	\subsection{Frequency Functionalities}\label{pvgroup:function} %LABEL NOT CHANGED YET

	\paragraph{}

		\begin{tabular}{N}
			\hline
			\bfseries FreqStartFreq-SP \\ \hline
			\emph{Start Frequency Set Point} \\
			Data type: float \\
			Unit: Hz \\
			Range: 9 kHz to 3 GHz \\
			Description: Sets the start frequency for the RF sweep. This parameter relates to the center frequency and span. If you change the frequency, these parameters change accordingly. \\
			Command: FREQ:STAR \emph{value} \\
			\\
			
		\end{tabular}


		\begin{tabular}{N}
			\hline
			\bfseries FreqStartFreq-RB \\ \hline
			\emph{Start Frequency Read Back} \\
			Data type: float \\
			Unit: Hz \\
			Description: Reads the start frequency for the RF sweep. \\
			Command: FREQ:STAR? \\
			\\

		\end{tabular}


		\begin{tabular}{N}
			\hline
			\bfseries FreqFStepLin-SP \\ \hline
			\emph{Linear Step Size Set Point} \\
			Data type: float \\
			Unit: Hz \\
			Range: 9 kHz to 3 GHz \\
			Description: Sets the step size for linear RF frequency sweep steps. This parameter is related to the number of steps (SOUR:SWE:FREQ:POIN) within the sweep range. \\
			Command: SWE:FREQ:STEP:LIN \emph{value} \\
			\\
			
		\end{tabular}


		\begin{tabular}{N}
			\hline
			\bfseries FreqFStepLin-RB \\ \hline
			\emph{Linear Step Size Read Back} \\
			Data type: float \\
			Unit: Hz \\
			Description: Reads the step size for linear RF frequency sweep. \\
			Command: SWE:FREQ:STEP:LIN? \\
			\\

		\end{tabular}


		\begin{tabular}{N}
			\hline
			\bfseries FreqStopFreq-SP \\ \hline
			\emph{Stop Frequency Set Point} \\
			Data type: float \\
			Unit: Hz \\
			Range: 9 kHz to 3 GHz \\
			Description: Sets the stop frequency for the RF sweep.
This parameter is related to the center frequency and span. If you change the fre-
quency, these parameters change accordingly. \\
			Command: FREQ:STOP \emph{value} \\
			\\
			
		\end{tabular}


		\begin{tabular}{N}
			\hline
			\bfseries FreqStopFreq-RB \\ \hline
			\emph{Stop Frequency Read Back} \\
			Data type: float \\
			Unit: Hz \\
			Description: Reads the stop frequency for the RF sweep. \\
			Command: FREQ:STOP? \\
			\\

		\end{tabular}


		\begin{tabular}{N}
			\hline
			\bfseries FreqCenterFreq-SP \\ \hline
			\emph{Center Frequency Set Point} \\
			Data type: float \\
			Unit: Hz \\
			Range: 9 kHz to 3 GHz \\
			Description: Sets the center frequency of the RF sweep range. The range is defined by this center frequency and the specified Frequency Span (SOUR:FREQ:SPAN). \\
			Command: SOUR:FREQ:CENT \emph{value} \\
			\\
			
		\end{tabular}


		\begin{tabular}{N}
			\hline
			\bfseries FreqCenterFreq-RB \\ \hline
			\emph{Center Frequency Read Back} \\
			Data type: float \\
			Unit: Hz \\
			Description: Reads the center frequency for the RF frequency sweep. \\
			Command: SOUR:FREQ:CENT? \\
			\\

		\end{tabular}


		\begin{tabular}{N}
			\hline
			\bfseries FreqFreqMan-SP \\ \hline
			\emph{Frequency Manually Set Point} \\
			Data type: float \\
			Unit: Hz \\
			Range: 9 kHz to 3 GHz \\
			Description: Determines the frequency and triggers a sweep step manually, when in SWE:MODE MAN. \\
			Command: SOUR:FREQ:MAN \emph{value} \\
			\\
			
		\end{tabular}


		\begin{tabular}{N}
			\hline
			\bfseries FreqFreqMan-RB \\ \hline
			\emph{Frequency Manually Written Read Back} \\
			Data type: float \\
			Unit: Hz \\
			Description: Reads the frequency manually written. \\
			Command: SOUR:FREQ:MAN? \\
			\\

		\end{tabular}


		\begin{tabular}{N}
			\hline
			\bfseries FreqFreqSpan-SP \\ \hline
			\emph{Frequency Span Set Point} \\
			Data type: float \\
			Unit: Hz \\
			Range: 9 kHz to 3 GHz \\
			Description: Determines the extent of the frequency sweep range. This setting in combination with the Center Frequency setting (SOUR:FREQ:CENT) defines the sweep range. \\
			Command: SOUR:FREQ:SPAN \emph{value} \\
			\\
			
		\end{tabular}


		\begin{tabular}{N}
			\hline
			\bfseries FreqFreqSpan-RB \\ \hline
			\emph{Frequency Span Read Back} \\
			Data type: float \\
			Unit: Hz \\
			Description: Reads the extent of the frequency sweep range. \\
			Command: SOUR:FREQ:SPAN? \\
			\\

		\end{tabular}


		\begin{tabular}{N}
			\hline
			\bfseries FreqFExeSweep-Cmd \\ \hline
			\emph{Execute Frequency Sweep Command} \\
			Data type: bool \{\begin{itemize}[noitemsep]
				\small
				\item[] 0
				\item[] 1
			\end{itemize}\} \\
			Description: Starts an RF frequency sweep cycle, manually. This command is only effective in single mode. \\
			Command: SOUR:SWE:FREQ:EXEC \\
			\\

		\end{tabular}


		\begin{tabular}{N}
			\hline
			\bfseries FreqFSweepMode-Sel \\ \hline
			\emph{Frequency Sweep Mode Selection} \\
			Data type: enum \\
			Description: Sets the sweep mode.\begin{itemize}[noitemsep]
				\small
				\item[] \textbf{AUTO}
				\item[] Each trigger triggers exactly one complete sweep.
				\item[] \textbf{MANual}
				\item[] The trigger system is not active. Each frequency step of the sweep is triggered individually by means of a FREQ:MAN command under remote control. The frequency is set directly with the command FREQ:MAN.
				\item[] \textbf{STEP}
				\item[] Each trigger triggers one sweep step only (Single Step Mode). The frequency increases by the value entered under Linear Spacing (SOUR:SWE:FREQ:STEP:LIN) ou Logarithmic Spacing (SOUR:SWE:FREQ:STEP:LOG).
			\end{itemize} \\
			Command: SOUR:SWE:FREQ:MODE \emph{value} \\
			\\

		\end{tabular}


		\begin{tabular}{N}
			\hline
			\bfseries FreqFSweepMode-Sts \\ \hline
			\emph{Frequency Sweep Mode Status} \\
			Data type: enum \{\begin{itemize}[noitemsep]
				\small
				\item[] AUTO
				\item[] MAN
				\item[] STEP
			\end{itemize}\} \\
			Description: Shows the frequency sweep mode. \\
			Command: SOUR:SWE:FREQ:MODE? \\
			\\

		\end{tabular}


		\begin{tabular}{N}
			\hline
			\bfseries FreqFSweepPts-SP \\ \hline
			\emph{Number of Frequency Sweep Steps Set Point} \\
			Data type: integer \\
			Range: 2 to Max
			Description: Determines the number of steps for the RF frequency sweep within the sweep range. This parameter always applies to the currently set sweep spacing (Linear or Logarithmic) and correlates with the step size. \\
			Command: SOUR:SWE:FREQ:POIN \emph{value} \\
			Configuration needed: \begin{itemize}[noitemsep]
				\small
				\item[] SOUR:SWE:FREQ:MODE MAN
			\end{itemize} \\
		
		\end{tabular}


		\begin{tabular}{N}
			\hline
			\bfseries FreqFSweepPts-RB \\ \hline
			\emph{Number of Frequency Sweep Steps Read Back} \\
			Data type: integer \\
			Description: Read number of steps for the RF frequency sweep. \\
			Command: SOUR:SWE:FREQ:POIN? \\
			\\

		\end{tabular}


		\begin{tabular}{N}
			\hline
			\bfseries FreqFreqRetr-Sel \\ \hline
			\emph{Retrace State for the Frequency Sweep Enable Selection} \\
			Data type: bool \{\begin{itemize}[noitemsep]
				\small
				\item[] OFF
				\item[] ON
			\end{itemize}\} \\
			Description: Selects if the signal changes to the start frequency value while it is waiting for the next trigger event. You can enable this feature, when you are working with sawtooth shapes in sweep mode "Single" or "External Single". \\
			Command: SOUR:SWE:FREQ:RETR \emph{value} \\
			Configuration needed: \begin{itemize}[noitemsep]
				\small
				\item[] SOUR:FREQ:MODE SWE
				\item[] SOUR:SWE:FREQ:MODE AUTO
				\item[] TRIG:FSW:SOUR MAN|EXT
			\end{itemize} \\

		\end{tabular}


		\begin{tabular}{N}
			\hline
			\bfseries FreqFreqRetr-Sts \\ \hline
			\emph{Retrace State for the Frequency Sweep Enable Status} \\
			Data type: bool \{\begin{itemize}[noitemsep]
				\small
				\item[] OFF
				\item[] ON
			\end{itemize}\} \\
			Description: Shows if the signal changes to the start frequency value while it is waiting for the next trigger event. \\
			Command: SOUR:SWE:FREQ:RETR? \\
			\\

		\end{tabular}


		\begin{tabular}{N}
			\hline
			\bfseries FreqFRunnMode-Mon \\ \hline
			\emph{Frequency Sweep Running Mode Monitor} \\
			Data type: bool \{\begin{itemize}[noitemsep]
				\small
				\item[] OFF
				\item[] ON
			\end{itemize}\} \\
			Description: Queries the current state of the frequency sweep mode. \\
			Command: SOUR:SWE:FREQ:RUNN? \\
			\\

		\end{tabular}


		\begin{tabular}{N}
			\hline
			\bfseries FreqFreqShp-Sel \\ \hline
			\emph{Sequence Shape for Level Sweep Selection} \\
			Data type: enum \\
			Description: Sets the wave shape for the frequency sweep.\begin{itemize}[noitemsep]
				\small
				\item[] \textbf{SAWTooth}
				\item[] The shape of the sweep sequence resembles a sawtooth. One sweep runs from start to stop frequency. Each subsequent sweep starts at the start frequency.
				\item[] \textbf{TRIangle}
				\item[] The shape of the sweep resembles a triangle. One sweep runs from start to stop frequency and back. Each subsequent sweep starts at the start frequency.
			\end{itemize} \\
			Command: SOUR:SWE:FREQ:MODE \emph{value} \\
			\\

		\end{tabular}


		\begin{tabular}{N}
			\hline
			\bfseries FreqFreqShp-Sts \\ \hline
			\emph{Sequence Shape Frequency Sweep Status} \\
			Data type: enum \{\begin{itemize}[noitemsep]
				\small
				\item[] SAWT
				\item[] TRI
			\end{itemize}\} \\
			Description: Shows the frequency sweep wave shape. \\
			Command: SOUR:SWE:FREQ:MODE? \\
			\\

		\end{tabular}


		\begin{tabular}{N}
			\hline
			\bfseries FreqFStepLog-SP \\ \hline
			\emph{Logarithmic Step for Frequency Sweep Set Point} \\
			Data type: float \\
			Unit: \% \\
			Range: 9 kHz to 3 GHz \\
			Description: Sets a logarithmically determined sweep step size for the RF frequency sweep. It is expressed in percent and you must enter the value and the unit PCT with the command. The frequency is increased by a logarithmically calculated fraction of the current frequency. \\
			Command: SOUR:SWE:FREQ:STEP:LOG \emph{value} \\
			\\
			
		\end{tabular}


		\begin{tabular}{N}
			\hline
			\bfseries FreqFStepLog-RB \\ \hline
			\emph{Logarithmic Step for Frequency Sweep Read Back} \\
			Data type: float \\
			Unit: \% \\
			Description: Reads a logarithmically determined sweep step size for the RF frequency sweep. \\
			Command: SOUR:SWE:FREQ:STEP:LOG? \\
			\\

		\end{tabular}


		\begin{tabular}{N}
			\hline
			\bfseries FreqFDwellTime-SP \\ \hline
			\emph{Dwell Time for Frequency Sweep Set Point} \\
			Data type: float \\
			Unit: s \\
			Range: 2 ms to 100 s \\
			Description: Sets the time taken for each frequency step of the sweep. It is recommended to switch off the "Display Update" for optimum sweep performance especially with short dwell times (SYSTem:DISPlay:UPDate OFF). \\
			Command: SOUR:SWE:FREQ:DWEL \emph{value} \\
			\\
			
		\end{tabular}


		\begin{tabular}{N}
			\hline
			\bfseries FreqFDwellTime-RB \\ \hline
			\emph{Dwell Time for Frequency Sweep Read Back} \\
			Data type: float \\
			Unit: s \\
			Description: Reads time taken for each frequency step of the sweep. \\
			Command: SOUR:SWE:FREQ:DWEL? \\
			\\

		\end{tabular}


		\begin{tabular}{N}
			\hline
			\bfseries FreqFSpacMode-Sel \\ \hline
			\emph{Spacing Mode for the Frequency Sweep Selection} \\
			Data type: enum \\
			Description: Selects the mode for the calculation of the frequency sweep intervals. The frequency increases or decreases by this value at each step.\begin{itemize}[noitemsep]
				\small
				\item[] \textbf{LINear}
				\item[] With the linear sweep, the step width is a fixed frequency value which is added to the current frequency.
				\item[] \textbf{LOGarithmic}
				\item[] With the logarithmic sweep, the step width is a constant fraction of the current frequency. This fraction is added to the current frequency.
			\end{itemize} \\
			Command: SOUR:SWE:FREQ:SPAC \emph{value} \\
			\\

		\end{tabular}


		\begin{tabular}{N}
			\hline
			\bfseries FreqFSpacMode-Sts \\ \hline
			\emph{Spacing Mode for the Frequency Sweep Status} \\
			Data type: enum \{\begin{itemize}[noitemsep]
				\small
				\item[] LIN
				\item[] LOG
			\end{itemize}\} \\ 
			Description: Shows the the spacing mode for the frequency sweep. \\
			Command: SOUR:SWE:FREQ:SPAC? \\
			\\

		\end{tabular}


		\begin{tabular}{N}
			\hline
			\bfseries FreqVarMode-Sel \\ \hline
			\emph{Variation Mode for the Frequency Selection} \\
			Data type: enum \{\begin{itemize}[noitemsep]
				\small
				\item[] DECimal (OFF)
				\item[] USER (ON)
			\end{itemize}\} \\
			Description:Activates (USER) or deactivates (DECimal) the user-defined step width used when varying the frequency value. The command is linked to the command "Variation Active" for manual control. \\
			Command: SOUR:FREQ:STEP:MODE \emph{value} \\
			\\

		\end{tabular}


		\begin{tabular}{N}
			\hline
			\bfseries FreqVarMode-Sts \\ \hline
			\emph{Variation Mode for the Frequency Status} \\
			Data type: enum \{\begin{itemize}[noitemsep]
				\small
				\item[] DEC
				\item[] USER
			\end{itemize}\} \\ 
			Description: Shows the user-defines step width used when varying the frequency value. \\
			Command: SOUR:FREQ:STEP:MODE? \\
			\\

		\end{tabular}


		\begin{tabular}{N}
			\hline
			\bfseries FreqStepVar-SP \\ \hline
			\emph{Step Variation for Frequency Sweep Set Point} \\
			Data type: float \\
			Unit: Hz \\
			Range: 9 kHz to 3 GHz \\
			Description: Sets the step width for the frequency.\\
			Command: SOUR:FREQ:STEP:INCR \emph{value} \\
			Configuration needed: \begin{itemize}[noitemsep]
                                 \small
                                 \item[] SOUR:FREQ:STEP:MODE USER
                         \end{itemize} \\
			
		\end{tabular}


		\begin{tabular}{N}
			\hline
			\bfseries FreqStepVar-RB \\ \hline
			\emph{Step Variation for Frequency Sweep Read Back} \\
			Data type: float \\
			Unit: Hz \\
			Description: Reads the step width for the frequency. \\
			Command: SOUR:FREQ:STEP:INCR? \\
			\\

		\end{tabular}


		\begin{tabular}{N}
			\hline
			\bfseries FreqPhsCont-Sel \\ \hline
			\emph{Retrace State for the Frequency Sweep Enable Selection} \\
			Data type: bool \{\begin{itemize}[noitemsep]
				\small
				\item[] OFF
				\item[] ON
			\end{itemize}\} \\
			Description: Activates/deactivates phase continuous frequency settings. For a given RF frequency setting, phase continuous frequency changes are possible in a limited frequency range. The output sinewave is phase continuous. \\
			Command: SOUR:FREQ:PHAS:CONT:STAT \emph{value} \\
			\\

		\end{tabular}


		\begin{tabular}{N}
			\hline
			\bfseries FreqPhsCont-Sts \\ \hline
			\emph{Retrace State for the Frequency Sweep Enable Status} \\
			Data type: bool \{\begin{itemize}[noitemsep]
				\small
				\item[] OFF
				\item[] ON
			\end{itemize}\} \\
			Description: Shows if the signal changes to the start frequency value while it is waiting for the next trigger event. \\
			Command: SOUR:FREQ:PHAS:CONT:STAT? \\
			\\

		\end{tabular}


		\begin{tabular}{N}
			\hline
			\bfseries FreqRange-Sel \\ \hline
			\emph{Frequency Range Mode Selection} \\
			Data type: enum \\
			Description: Selects the mode for determining the frequency range for the phase continuous signal. The minimum (SOUR:FREQ:PHAS:CONT:LOW) and maximum (SOUR:FREQ:PHAS:CONT:HIGH) frequency of the frequency range depends on the mode selected with this command.\begin{itemize}[noitemsep]
				\small
				\item[] \textbf{NARRow}
				\item[] The available frequency range is smaller than with setting wide. It is asymmetrical around the RF frequency set at the point of activating the phase continuous settings.
				\item[] \textbf{WIDE}
				\item[] The wide mode provides a larger frequency range. The frequency range is symmetrical around the RF frequency set at the point of activating the phase continuous settings.
			\end{itemize} \\
			Command: SOUR:FREQ:PHAS:CONT:MODE \emph{value} \\
			\\

		\end{tabular}


		\begin{tabular}{N}
			\hline
			\bfseries FreqRange-Sts \\ \hline
			\emph{Frequency Range Mode Status} \\
			Data type: enum \{\begin{itemize}[noitemsep]
				\small
				\item[] NARRow
				\item[] WIDE
			\end{itemize}\} \\ 
			Description: Shows the mode for determining the frequency range for the phase continuous signal. \\
			Command: SOUR:FREQ:PHAS:CONT:MODE? \\
			\\

		\end{tabular}


		\begin{tabular}{N}
			\hline
			\bfseries FreqContPhsHi-Mon \\ \hline
			\emph{Maximum Frequency for Frequency Range Monitor} \\
			Data type: float \\
			Unit: Hz \\
			Description: Queries the maximum frequency of the frequency range for phase continuous settings. The maximum frequency of the frequency range depends on the mode selected by the PV FreqRange-Sel.\\
			Command: SOUR:FREQ:PHAS:CONT:HIGH \emph{value} \\
			\\
			
		\end{tabular}


		\begin{tabular}{N}
			\hline
			\bfseries FreqContPhsLo-Mon \\ \hline
			\emph{Minimum Frequency for Frequency Range Monitor} \\
			Data type: float \\
			Unit: Hz \\
			Description: Queries the minimum frequency of the frequency range for phase continuous settings. The minimum frequency of the frequency range depends on the mode selected by the PV FreqRange-Sel.\\
			Command: SOUR:FREQ:PHAS:CONT:HIGH \emph{value} \\
			\\
			
		\end{tabular}


		\begin{tabular}{N}
			\hline
			\bfseries FreqPDwellTime-SP \\ \hline
			\emph{Dwell Time for Level Sweep Set Point} \\
			Data type: float \\
			Unit: s \\
			Range: 1 ms to 100 s \\
			Description: Sets the time taken for each level step of the sweep. \\
			Command: SOUR:SWE:POW:DWEL \emph{value} \\
			\\
			
		\end{tabular}


		\begin{tabular}{N}
			\hline
			\bfseries FreqPDwellTime-RB \\ \hline
			\emph{Dwell Time for Level Sweep Read Back} \\
			Data type: float \\
			Unit: s \\
			Description: Reads the dwell time for each step of the sweep. \\
			Command: SOUR:SWE:POW:DWEL? \\
			\\

		\end{tabular}


		\begin{tabular}{N}
			\hline
			\bfseries FreqPSweepMode-Sel \\ \hline
			\emph{Level Sweep Mode Selection} \\
			Data type: enum \\
			Description: Selects the cycle mode of the level sweep.\begin{itemize}[noitemsep]
				\small
				\item[] \textbf{AUTO}
				\item[] Each trigger triggers exactly one complete sweep.
				\item[] \textbf{MANual}
				\item[] The trigger system is not active. Each level step of the sweep is triggered individually, by varying the "Current Level" value. The level increases by the value specified under SWEep:POW:STEP which each sent :POW:MAN command, irrespective the value entered there.
				\item[] \textbf{STEP}
				\item[] Each trigger triggers one sweep step only. The level increases by the value entered under SOUR:SWEep:POWer:STEP.
			\end{itemize} \\
			Command: SOUR:SWE:POW:MODE \emph{value} \\
			\\

		\end{tabular}


		\begin{tabular}{N}
			\hline
			\bfseries FreqPSweepMode-Sts \\ \hline
			\emph{Level Sweep Mode Status} \\
			Data type: enum \{\begin{itemize}[noitemsep]
				\small
				\item[] AUTO
				\item[] MAN
				\item[] STEP
			\end{itemize}\} \\ 
			Description: Shows the level sweep mode. \\
			Command: SOUR:SWE:POW:MODE? \\
			\\

		\end{tabular}


		\begin{tabular}{N}
			\hline
			\bfseries FreqPSweepPts-SP \\ \hline
			\emph{Number of Steps for Level Sweep Set Point} \\
			Data type: integer \\
			Unit: s \\
			Range: 2 to Max \\ 
			Description: Sets the number of steps for the RF level sweep within the sweep range. This parameter always applies to the currently set sweep spacing and correlates with the step size. If you change the number of sweep points, the step size changes accordingly. The sweep range remains the same. \\
			Command: SOUR:SWE:POW:POIN \emph{value} \\
			\\
			
		\end{tabular}


		\begin{tabular}{N}
			\hline
			\bfseries FreqPSweepPts-RB \\ \hline
			\emph{Number of Steps for Level Sweep Read Back} \\
			Data type: integer \\
			Unit: s \\
			Description: Reads the number of steps for the level sweep. \\
			Command: SOUR:SWE:POW:POIN? \\
			\\

		\end{tabular}


		\begin{tabular}{N}
			\hline
			\bfseries FreqLvlRetr-Sel \\ \hline
			\emph{Retrace State for the Level Sweep Selection} \\
			Data type: bool \{\begin{itemize}[noitemsep]
				\small
				\item[] OFF
				\item[] ON
			\end{itemize}\} \\
			Description: Activates that the signal changes to the start level value while it is waiting for the next trigger event. You can enable this feature, when you are working with sawtooth shapes in sweep mode "Single" or "External Single". \\
			Command: SOUR:SWE:POW:RETR \emph{value} \\
			Configuration needed:\begin{itemize}[noitemsep]
				\small
				\item[] CONFIGURAR SINGLE E EXT. SINGLE
			\end{itemize} \\

		\end{tabular}


		\begin{tabular}{N}
			\hline
			\bfseries FreqLvlRetr-Sts \\ \hline
			\emph{Retrace State for the Level Sweep Status} \\
			Data type: bool \{\begin{itemize}[noitemsep]
				\small
				\item[] OFF
				\item[] ON
			\end{itemize}\} \\
			Description: Shows if the signal changes to the start level value while it is waiting for the next trigger event. \\
			Command: SOUR:SWE:POW:RETR? \\
			\\
			
		\end{tabular}


		\begin{tabular}{N}
			\hline
			\bfseries FreqPRunnMode-Mon \\ \hline
			\emph{Level Sweep Monitor} \\
			Data type: bool \{\begin{itemize}[noitemsep]
				\small
				\item[] OFF
				\item[] ON
			\end{itemize}\} \\
			Description: Queries the current state of the level sweep mode. \\
			Command: SOUR:SWE:POW:RUNN? \\
			\\
			
		\end{tabular}


		\begin{tabular}{N}
			\hline
			\bfseries FreqLvlShp-Sel \\ \hline
			\emph{Sequence Shape for Level Sweep Selection} \\
			Data type: enum \\
			Description: Selects the cycle mode for a sweep sequence.\begin{itemize}[noitemsep]
				\small
				\item[] \textbf{SAWTooth} 
				\item[] One sweep runs from the start level to the stop level. The subsequent sweep starts at the start level again. The shape of sweep sequence resembles a sawtooth.
				\item[] \textbf{TRIangle}
				\item[] One sweep runs from start to stop level and back. The shape of the sweep resembles a triangle. Each subsequent sweep starts at the start level again.			
			\end{itemize} \\
			Command: SOUR:SWE:POW:SHAP \emph{value} \\
			\\

		\end{tabular}


		\begin{tabular}{N}
			\hline
			\bfseries FreqLvlShp-Sts \\ \hline
			\emph{Sequence Shape for Level Sweep Status} \\
			Data type: enum \{\begin{itemize}[noitemsep]
				\small
				\item[] SAWT
				\item[] TRI
			\end{itemize}\} \\ 
			Description: Shows the sequence shape for the level sweep. \\
			Command: SOUR:SWE:POW:SHAP? \\
			\\

		\end{tabular}


		\begin{tabular}{N}
			\hline
			\bfseries FreqPSpacMode-Mon \\ \hline
			\emph{Spacing Mode for Level Sweep Monitor} \\
			Data type: enum \{\begin{itemize}[noitemsep]
				\small
				\item[] LINear
			\end{itemize}\} \\
			Description: Queries the sweep spacing mode. The sweep spacing for level sweeps is always linear. \\
			Command: SOUR:SWE:POW:SPAC:MODE? \\
			\\
			
		\end{tabular}


		\begin{tabular}{N}
			\hline
			\bfseries FreqPStepLog-SP \\ \hline
			\emph{Logarithmic Step Size for Level Sweep Set Point} \\
			Data type: float \\
			Unit: dB \\ 
			Description: Sets a logarithmically determined sweep step size for the RF level sweep. This parameter correlates with the number of steps (SOUR:SWE:POW:POIN) whtihin the sweep range. If the step size is changed, the number of steps changes accordingly. The sweep range remains the same. \\
			Command: SOUR:SWE:POW:STEP:LOG \emph{value} \\
			\\
			
		\end{tabular}


		\begin{tabular}{N}
			\hline
			\bfseries FreqPStepLog-RB \\ \hline
			\emph{Logarithmic Step Size for Level Sweep Read Back} \\
			Data type: float \\
			Unit: dB \\
			Description: Reads logarithmic step size for the level sweep. \\
			Command: SOUR:SWE:POW:STEP:LOG? \\
			\\

		\end{tabular}


		\begin{tabular}{N}
			\hline
			\bfseries FreqFreqMode-Sel \\ \hline
			\emph{RF Signal Frequency Mode Selection} \\
			Data type: enum \\
			Description: Selects the frequency mode for the generating RF output signal. The selected mode determines the parameters to be used for further frequency settings.\begin{itemize}[noitemsep]
				\small
				\item[] \textbf{CW} 
				\item[] Sets the fixed frequency mode.
				\item[] \textbf{SWEep}
				\item[] Sets the sweep mode. The instrument processes the frequency settings in defined sweep steps.
				\item[] \textbf{LIST}
				\item[] Sets the list mode. The instrument processes the frequency and level settings by means of values loaded from a list.
			\end{itemize} \\
			Command: FREQ:MODE \emph{value} \\
			\\

		\end{tabular}


		\begin{tabular}{N}
			\hline
			\bfseries FreqFreqMode-Sts \\ \hline
			\emph{RF Signal Frequency Mode Status} \\
			Data type: enum \{\begin{itemize}[noitemsep]
				\small
				\item[] CW
				\item[] SWE
				\item[] LIST
			\end{itemize}\} \\ 
			Description: Shows the frequency mode for the RF output signal. \\
			Command: FREQ:MODE? \\
			\\

		\end{tabular}


		\begin{tabular}{N}
			\hline
			\bfseries FreqLFSweepMode-Sel \\ \hline
			\emph{RF Signal Frequency Mode Selection} \\
			Data type: enum \\
			Description: Selects the cycle mode of the LF sweep.\begin{itemize}[noitemsep]
				\small
				\item[] \textbf{AUTO} 
				\item[] Performs a complete sweep cycle from the start to the end value when a trigger event occurs. The dwell time determines the time period for the signal to switch to the next step.
				\item[] \textbf{MANual}
				\item[] Performs a single sweep step when a manual trigger event occurs.
				\item[] \textbf{STEP}
				\item[] Each trigger triggers one sweep step only.
			\end{itemize} \\
			Command: SOUR:LFO:SWE:FREQ:MODE \emph{value} \\
			\\

		\end{tabular}


		\begin{tabular}{N}
			\hline
			\bfseries FreqLFSweepMode-Sts \\ \hline
			\emph{RF Signal Frequency Mode Status} \\
			Data type: enum \{\begin{itemize}[noitemsep]
				\small
				\item[] AUTO
				\item[] MAN
				\item[] STEP
			\end{itemize}\} \\ 
			Description: Reads the cycle mode of the LF sweep. \\
			Command: SOUR:LFO:SWE:FREQ:MODE? \\
			\\

		\end{tabular}


		\begin{tabular}{N}
			\hline
			\bfseries FreqLFMode-Sel \\ \hline
			\emph{Instruments Operating Mode Selection} \\
			Data type: enum \\
			Description: Sets the instrument operating mode, and determines the commands to be used for frequency settings.\begin{itemize}[noitemsep]
				\small
				\item[] \textbf{CW} 
				\item[] The instrument operates at a fixed frequency.
				\item[] \textbf{SWEep}
				\item[] Sets the sweep mode. The instrument processes the frequency settings in defined sweep steps.
			\end{itemize} \\
			Command: SOUR:LFO:FREQ:MODE \emph{value} \\
			\\

		\end{tabular}


		\begin{tabular}{N}
			\hline
			\bfseries FreqLFMode-Sts \\ \hline
			\emph{Instruments Operating Mode Status} \\
			Data type: enum \{\begin{itemize}[noitemsep]
				\small
				\item[] CW
				\item[] SWE
			\end{itemize}\} \\ 
			Description: Reads the instrument operating mode. \\
			Command: SOUR:LFO:FREQ:MODE? \\
			\\

		\end{tabular}


		\begin{tabular}{N}
			\hline
			\bfseries FreqLFSweepSrc-Sel \\ \hline
			\emph{LF Sweep Source Selection} \\
			Data type: enum \\
			Description: Selects the source for the LF sweep.\begin{itemize}[noitemsep]
				\small
				\item[] \textbf{LF1} 
				\item[] Selects LF generator 1.
				\item[] \textbf{LF2}
				\item[] Selects LF generator 2.
			\end{itemize} \\
			Command: SOUR:LFO:SWE:FREQ:LFS \emph{value} \\
			\\

		\end{tabular}


		\begin{tabular}{N}
			\hline
			\bfseries FreqLFSweepSrc-Sts \\ \hline
			\emph{LF Sweep Source Status} \\
			Data type: enum \{\begin{itemize}[noitemsep]
				\small
				\item[] LF1
				\item[] LF2
			\end{itemize}\} \\ 
			Description: Reads the source for the LF sweep. \\
			Command: SOUR:LFO:SWE:FREQ:LFS? \\
			\\

		\end{tabular}


		\begin{tabular}{N}
			\hline
			\bfseries FreqLFStartFreq-SP \\ \hline
			\emph{Start Frequency for LF Sweep Set Point} \\
			Data type: float \\
			Unit: Hz \\ 
			Range: 9 kHz to 3 GHz
			Description: Sets the start frequency for the LF sweeep.\\
			Command: SOUR:LFO:FREQ:STAR \emph{value} \\
			\\
			
		\end{tabular}


		\begin{tabular}{N}
			\hline
			\bfseries FreqLFStartFreq-RB \\ \hline
			\emph{Start Frequency for LF Sweep Read Back} \\
			Data type: float \\
			Unit: Hz \\
			Description: Reads the start frequency for the LF sweep. \\
			Command: SOUR:LFO:FREQ:STAR? \\
			\\

		\end{tabular}


		\begin{tabular}{N}
			\hline
			\bfseries FreqLFStopFreq-SP \\ \hline
			\emph{Stop Frequency for LF Sweep Set Point} \\
			Data type: float \\
			Unit: Hz \\ 
			Range: 9 kHz to 3 GHz
			Description: Sets the stop frequency for the LF sweeep.\\
			Command: SOUR:LFO:FREQ:STOP \emph{value} \\
			\\
			
		\end{tabular}


		\begin{tabular}{N}
			\hline
			\bfseries FreqLFStopFreq-RB \\ \hline
			\emph{Stop Frequency for LF Sweep Read Back} \\
			Data type: float \\
			Unit: Hz \\
			Description: Reads the stop frequency for the LF sweep. \\
			Command: SOUR:LFO:FREQ:STOP? \\
			\\

		\end{tabular}


		\begin{tabular}{N}
			\hline
			\bfseries FreqLFSpac-Sel \\ \hline
			\emph{Spacing Mode for LF Sweep Selection} \\
			Data type: enum \\ 
			Description: Selects the mode for the calculation of the frequency sweep intervals.The frequency increases or decreases by this value at each step.\begin{itemize}[noitemsep]
				\small
				\item[] \textbf{LINear} 
				\item[] With the linear sweep, the step width is a fixed frequency value which is added to the current frequency.
				\item[] \textbf{LOGarithmic}
				\item[] With the logarithmic sweep, the step width is a constant fraction of the current frequency. This fraction is added to the current frequency.
			\end{itemize} \\
			Command: SOUR:LFO:SWE:FREQ:SPAC \emph{value} \\
			\\

		\end{tabular}


		\begin{tabular}{N}
			\hline
			\bfseries FreqLFSpac-Sts \\ \hline
			\emph{Spacing Mode for LF Sweep Status} \\
			Data type: enum \{\begin{itemize}[noitemsep]
				\small
				\item[] LIN
				\item[] LOG
			\end{itemize}\} \\ 
			Description: Reads the spacing mode for the LF sweep. \\
			Command: SOUR:LFO:SWE:FREQ:SPAC? \\
			\\

		\end{tabular}


		\begin{tabular}{N}
			\hline
			\bfseries FreqLFShp-Sel \\ \hline
			\emph{Sequence Shape for LF Sweep Selection} \\
			Data type: enum \\ 
			Description: Sets the cycle mode for a sweep sequence (shape).\begin{itemize}[noitemsep]
				\small
				\item[] \textbf{SAWTooth} 
				\item[]	A sweep runs from the start to the stop frequency. A subsequent sweep starts at the start frequency, that menas the shape of the sweep sequence resembles a sawtooth.
				\item[] \textbf{TRIangle}
				\item[] A sweep runs from the start to the stop frequency and back, that means the shape of the sweep resembles a triangle. A subsequent sweep starts at the start frequency.
			\end{itemize} \\
			Command: SOUR:LFO:SWE:FREQ:SHAP \emph{value} \\
			\\

		\end{tabular}


		\begin{tabular}{N}
			\hline
			\bfseries FreqLFShp-Sts \\ \hline
			\emph{Sequence Shape for LF Sweep Status} \\
			Data type: enum \{\begin{itemize}[noitemsep]
				\small
				\item[] LIN
				\item[] LOG
			\end{itemize}\} \\ 
			Description: Reads the cycle mode for the LF sweep sequence. \\
			Command: SOUR:LFO:SWE:FREQ:SHAP? \\
			\\

		\end{tabular}


		\begin{tabular}{N}
			\hline
			\bfseries FreqLFStepLin-SP \\ \hline
			\emph{Linear Step Size for LF Sweep Set Point} \\
			Data type: float \\
			Unit: Hz \\ 
			Range: 9 kHz to 3 GHz
			Description: Sets the linear step size for the LF frequency sweep steps.\\
			Command: SOUR:LFO:SWE:FREQ:STEP:LIN \emph{value} \\
			\\
			
		\end{tabular}


		\begin{tabular}{N}
			\hline
			\bfseries FreqLFStepLin-RB \\ \hline
			\emph{Linear Step Size for LF Sweep Read Back} \\
			Data type: float \\
			Unit: Hz \\
			Description: Reads the linear step size for the LF sweep. \\
			Command: SOUR:LFO:SWE:FREQ:STEP:LIN? \\
			\\

		\end{tabular}


		\begin{tabular}{N}
			\hline
			\bfseries FreqLFStepLog-SP \\ \hline
			\emph{Logarithmic Step Size for LF Sweep Set Point} \\
			Data type: float \\
			Range: 0.01 to 100 \\
			Description: Sets the logarithmically determined sweep step size for the LF frequency sweep. It is expressed in percent and you must enter the value and the unit PCT with the command.\\
			Command: SOUR:LFO:SWE:FREQ:STEP:LOG \emph{value} \\
			\\
			
		\end{tabular}


		\begin{tabular}{N}
			\hline
			\bfseries FreqLFStepLog-RB \\ \hline
			\emph{Logarithmic Step Size for LF Sweep Read Back} \\
			Data type: float \\
			Description: Reads the logarithmic step size for linear LF sweep. \\
			Command: SOUR:LFO:SWE:FREQ:STEP:LOG? \\
			\\

		\end{tabular}


		\begin{tabular}{N}
			\hline
			\bfseries FreqLFDwellTime-SP \\ \hline
			\emph{Dwell Time for LF Sweep Set Point} \\
			Data type: float \\
			Unit: s \\
			Range: 0.01 to 100 \\
			Description: Sets the dwell time for each frequency step of the sweep.\\
			Command: SOUR:LFO:SWE:FREQ:DWEL \emph{value} \\
			\\
			
		\end{tabular}


		\begin{tabular}{N}
			\hline
			\bfseries FreqLFDwellTime-RB \\ \hline
			\emph{Dwell Time for LF Sweep Read Back} \\
			Data type: float \\
			Unit: s \\
			Description: Reads the dwell time for each frequency step of the sweep. \\
			Command: SOUR:LFO:SWE:FREQ:DWEL? \\
			\\

		\end{tabular}


		\begin{tabular}{N}
			\hline
			\bfseries FreqPExeSweep-Cmd \\ \hline
			\emph{Execute Single Level Sweep Command} \\
			Data type: bool \{\begin{itemize}[noitemsep]
				\small
				\item[] OFF
				\item[] ON
			\end{itemize}\} \\
			Description:  \\
			Command: SOUR:SWE:POW:EXEC \emph{value} \\
			\\

		\end{tabular}


		\begin{tabular}{N}
			\hline
			\bfseries FreqRst-Cmd \\ \hline
			\emph{Reset All Active Sweeps Command} \\
			Data type: bool \{\begin{itemize}[noitemsep]
				\small
				\item[] OFF
				\item[] ON
			\end{itemize}\} \\
			Description: Reset all active sweeps to the starting point. \\
			Command: SOUR:SWE:RES:ALL \emph{value} \\
			\\

		\end{tabular}



	% TABLE: Modulation Functionalities
	\subsection{Modulation Functionalities}\label{pvgroup:function} %LABEL NOT CHANGED YET

		\paragraph{} % This paragraph aligns the first tabular with the others


		\begin{tabular}{N}
			\hline
			\bfseries ModAMSrc-Sel \\ \hline
			\emph{Signal Source for Amplitude Modulation Selection} \\
			Data type: enum \\  
			Description: Selects the modulation signal source for amplitude modulation. You can use both, the internal and an external modulation signal at a time.\begin{itemize}[noitemsep]
				\small
				\item[] \textbf{INTernal} 
				\item[]	Uses the internally generated signal for modulation.
				\item[] \textbf{EXTernal}
				\item[] Uses an externally applied modulation signal.
				\item[] \textbf{INT,EXT}
				\item[] Uses both, the internal and external modulation signals.
			\end{itemize} \\
			Command: SOUR:AM:SOUR \emph{value} \\
			\\

		\end{tabular}


		\begin{tabular}{N}
			\hline
			\bfseries ModAMSrc-Sts \\ \hline
			\emph{Signal Source for Amplitude Modulation Status} \\
			Data type: enum \{\begin{itemize}[noitemsep]
				\small
				\item[] INT
				\item[] EXT
				\item[] INT,EXT
			\end{itemize}\} \\ 
			Description: Reads the modulation signal source for the amplitude modulation. \\
			Command: SOUR:AM:SOUR? \\
			\\

		\end{tabular}


		\begin{tabular}{N}
			\hline
			\bfseries ModAM-Sel \\ \hline
			\emph{Amplitude Modulation Enable Selection} \\
			Data type: bool \{\begin{itemize}[noitemsep]
				\small
				\item[] OFF
				\item[] ON
			\end{itemize}\} \\
			Description: Activates amplitude modulation\\
			Command: SOUR:AM:STAT \emph{value} \\
			\\

		\end{tabular}


		\begin{tabular}{N}
			\hline
			\bfseries ModAM-Sts \\ \hline
			\emph{Amplitude Modulation Enable Status} \\
			Data type: bool \{\begin{itemize}[noitemsep]
				\small
				\item[] OFF
				\item[] ON
			\end{itemize}\} \\
			Description: Gets amplitude modulation state. \\
			Command: SOUR:AM:STAT? \\
			\\
			
		\end{tabular}


		\begin{tabular}{N}
			\hline
			\bfseries ModAMDepth-SP \\ \hline
			\emph{Amplitude Modulation Depth Set Point} \\
			Data type: float \\
			Unit: \% \\
			Range: 0 to 100 \\
			Description: Sets the modulation depth of the amplitude modulation signal in percent.\
			Command: SOUR:AM:DEPT \emph{value} \\
			\\
			
		\end{tabular}


		\begin{tabular}{N}
			\hline
			\bfseries ModAMDepth-RB \\ \hline
			\emph{Amplitude Modulation Depth Read Back} \\
			Data type: float \\
			Unit: \% \\
			Description: Reads the amplitude modulation depth. \\
			Command: SOUR:AM:DEPT? \\
			\\

		\end{tabular}


		\begin{tabular}{N}
			\hline
			\bfseries ModAMExtCoup-Sel \\ \hline
			\emph{AM Signal Coupling Mode Selection} \\
			Data type: enum \\  
			Description: Selects the coupling mode for the external amplitude modulation signal.\begin{itemize}[noitemsep]
				\small
				\item[] \textbf{AC} 
				\item[]	Uses only the AC signal component of the modulation signal.
				\item[] \textbf{DC}
				\item[] Uses the modulation signal as it is, with AC and DC.
			\end{itemize} \\
			Command: SOUR:AM:SOUR \emph{value} \\
			\\

		\end{tabular}


		\begin{tabular}{N}
			\hline
			\bfseries ModAMExtCoup-Sts \\ \hline
			\emph{AM Signal Coupling Mode Status} \\
			Data type: enum \{\begin{itemize}[noitemsep]
				\small
				\item[] AC
				\item[] DC
			\end{itemize}\} \\ 
			Description: Reads the coupling mode for the external amplitude modulation signal. \\
			Command: SOUR:AM:SOUR? \\
			\\

		\end{tabular}


		\begin{tabular}{N}
			\hline
			\bfseries ModAMIntDepth-SP \\ \hline
			\emph{Amplitude Modulation Internal Depth Set Point} \\
			Data type: float \\
			Unit: \% \\
			Range: 0 to dynamic \\
			Description: Sets the depth of the internal amplitude modulation signal in Hz. The sum of the deviations of all active frequency modulation signals may not exceed the total value set with SOUR:AM:DEPT. \\
			Command: SOUR:AM:INT:DEPT \emph{value} \\
			\\
			
		\end{tabular}


		\begin{tabular}{N}
			\hline
			\bfseries ModAMIntDepth-RB \\ \hline
			\emph{Amplitude Modulation Internal Depth Read Back} \\
			Data type: float \\
			Unit: \% \\
			Description: Reads the amplitude modulation internal depth. \\
			Command: SOUR:AM:INT:DEPT? \\
			\\

		\end{tabular}


		\begin{tabular}{N}
			\hline
			\bfseries ModAMSenS-Mon \\ \hline
			\emph{Sensitivity Monitor} \\
			Data type: float \\
			Unit: \%\\V \\
			Description: Monitors the amplitude modulation sensitivity. \\
			Command: SOUR:AM:SENS? \\
			\\

		\end{tabular}


		\begin{tabular}{N}
			\hline
			\bfseries ModAMIntSrc-Sel \\ \hline
			\emph{Amplitude Modulation Internal Source Selection} \\
			Data type: enum \\  
			Description: Selects the internal modulation signal source.\begin{itemize}[noitemsep]
				\small
				\item[] \textbf{NONE}
				\item[] Switches off all internal modulation sources.
				\item[] \textbf{LF1}
				\item[]	Internal LF generator 1.
				\item[] \textbf{LF2} 
				\item[]	Internal LF generator 2.
				\item[] \textbf{LF12} 
				\item[]	Selects both internal generators. LF frequency and modulation depth can be set separately. The added modulation depths of the two modulation generators must not exceed the overall modulation depth. This selection enables two-tone AM modulation.
				\item[] \textbf{NOISe} 
				\item[]	Selects noise signal. The modulation signal is white noise either with Gaussian distribution or equal distribution. This setting affects all analog modulations which use the noise generator as the internal modulation source.
				\item[] \textbf{LF1Nois} 
				\item[]	Internal LF generator 1 and the noise signal. In addition to the AM modulation signal, white noise is used as a modulation signal.
				\item[] \textbf{LF2Nois} 
				\item[]	Internal LF generator 2 and the noise signal. In addition to the AM modulation signal, white noise is used as a modulation signal.

			\end{itemize} \\
			Command: SOUR:AM:SOUR \emph{value} \\
			\\

		\end{tabular}


		\begin{tabular}{N}
			\hline
			\bfseries ModAMIntSrc-Sts \\ \hline
			\emph{Amplitude Modulation Internal Source Status} \\
			Data type: enum \{\begin{itemize}[noitemsep]
				\small
				\item[] NONE
				\item[] LF1
				\item[] LF2
				\item[] LF12
				\item[] NOIS
				\item[] LF1N
				\item[] LF2N
			\end{itemize}\} \\ 
			Description: Reads the coupling mode for the external amplitude modulation signal. \\
			Command: SOUR:AM:SOUR? \\
			\\

		\end{tabular}
%

		\begin{tabular}{N}
			\hline
			\bfseries ModFMSrc-Sel \\ \hline
			\emph{Frequency Modulation Source Selection} \\
			Data type: enum \\  
			Description: Selects the modulation signal source for frequency modulation. You can use both, the internal and an external modulation signal at a time.\begin{itemize}[noitemsep]
				\small
				\item[] \textbf{INTernal}
				\item[] Uses the internally generated signal for modulation.
				\item[] \textbf{EXTernal}
				\item[]	Uses an externally applied modulation signal. The external analog signal is input at the FM/PM EXT connector. The external digital signal is input at the AUX I/O connector (selection EDIGital).
				\item[] \textbf{INT,EXT} 
				\item[]	Uses both, the internal and external modulation signals.
				\item[] \textbf{EDIG} 
				\item[]	Uses an externally applied digital modulation signal.
			\end{itemize} \\
			Command: SOUR:FM:SOUR \emph{value} \\
			\\

		\end{tabular}


		\begin{tabular}{N}
			\hline
			\bfseries ModFMSrc-Sts \\ \hline
			\emph{Frequency Modulation Source Status} \\
			Data type: enum \{\begin{itemize}[noitemsep]
				\small
				\item[] INT
				\item[] EXT
				\item[] INT,EXT
				\item[] EDIG
			\end{itemize}\} \\ 
			Description: Reads the modulation signal source for the frquency modulation. \\
			Command: SOUR:FM:SOUR? \\
			\\

		\end{tabular}


		\begin{tabular}{N}
			\hline
			\bfseries ModFMExtCoup-Sel \\ \hline
			\emph{External FM Coupling Mode Selection} \\
			Data type: enum \\  
			Description: Selects the coupling mode for the external frequency modulation signal.\begin{itemize}[noitemsep]
				\small
				\item[] \textbf{AC}
				\item[] Uses only the AC signal component of the modulation signal.
				\item[] \textbf{DC}
				\item[] Uses the modulation signal as it is, with AC and DC.	
			\end{itemize} \\
			Command: SOUR:FM:EXT:COUP \emph{value} \\
			\\

		\end{tabular}


		\begin{tabular}{N}
			\hline
			\bfseries ModFMExtCoup-Sts \\ \hline
			\emph{External FM Coupling Source Status} \\
			Data type: enum \{\begin{itemize}[noitemsep]
				\small
				\item[] AC
				\item[] DC
			\end{itemize}\} \\ 
			Description: Reads the coupling mode for the external frequency modulation signal. \\
			Command: SOUR:FM:EXT:COUP? \\
			\\

		\end{tabular}


		\begin{tabular}{N}
			\hline
			\bfseries ModFMExtDev-SP \\ \hline
			\emph{External FM Deviation Set Point} \\
			Data type: float \\
			Unit: Hz \\ 
			Description: Sets the deviation of the external frequency modulation signal in Hz. The maximum deviation depends on the set RF frequency and the selected modulation mode (see data sheet). The sum of the deviations of all active frequency modulation signals may not exceed the total deviation value. \\
			Command: SOUR:FM:EXT:DEV \emph{value} \\
			\\
			
		\end{tabular}


		\begin{tabular}{N}
			\hline
			\bfseries ModFMExtDev-RB \\ \hline
			\emph{External FM Deviation Read Back} \\
			Data type: float \\
			Unit: Hz \\
			Description: Reads the external frequency modulation deviation. \\
			Command: SOUR:FM:EXT:DEV? \\
			\\

		\end{tabular}


		\begin{tabular}{N}
			\hline
			\bfseries ModFMIntDev-SP \\ \hline
			\emph{Internal FM Deviation Set Point} \\
			Data type: float \\
			Unit: Hz \\ 
			Description: Sets the deviation of the internal frequency modulation signals in Hz. The sum of the deviations of all active frequency modulation signals may not exceed the total deviation value. \\
			Command: SOUR:FM:INT:DEV \emph{value} \\
			\\
			
		\end{tabular}


		\begin{tabular}{N}
			\hline
			\bfseries ModFMIntDev-RB \\ \hline
			\emph{Internal FM Deviation Read Back} \\
			Data type: float \\
			Unit: Hz \\
			Description: Reads the internal frequency modulation deviation. \\
			Command: SOUR:FM:INT:DEV? \\
			\\

		\end{tabular}


		\begin{tabular}{N}
			\hline
			\bfseries ModFMIntSrc-Sel \\ \hline
			\emph{Internal FM Source Selection} \\
			Data type: enum \\  
			Description: Selects the internal frequency modulation signal source.\begin{itemize}[noitemsep]
				\small
				\item[] \textbf{NONE}
                                \item[] Switches off all internal modulation source     s.
                                \item[] \textbf{LF1}
                                \item[] Internal LF generator 1.
                                \item[] \textbf{LF2}
                                \item[] Internal LF generator 2.
                                \item[] \textbf{LF12}
                                \item[] Selects both internal generators. LF freque     ncy and modulation depth can be set separately. The added modulation depths of the two modulation generators must not exceed the overall modulation depth. This selection enables two-tone AM modulation.
                                \item[] \textbf{NOISe}
                                \item[] Selects noise signal. The modulation signal      is white noise either with Gaussian distribution or equal distribution. This setting affects all analog modulations which use the noise generator as the internal modulation source.
                                \item[] \textbf{LF1Nois}
                                \item[] Internal LF generator 1 and the noise signal. In addition to the AM modulation signal, white noise is used as a modulation signal.
                                \item[] \textbf{LF2Nois}
                                \item[] Internal LF generator 2 and the noise signal. In addition to the AM modulation signal, white noise is used as a modulation signal.
			\end{itemize} \\
			Command: SOUR:FM:INT:SOUR \emph{value} \\
			\\

		\end{tabular}


		\begin{tabular}{N}
			\hline
			\bfseries ModFMIntSrc-Sts \\ \hline
			\emph{Internal FM Source Status} \\
			Data type: enum \{\begin{itemize}[noitemsep]
				\small
				\item[] NONE
				\item[] LF1
				\item[] LF2
				\item[] LF12
				\item[] NOIS
				\item[] LF1N
				\item[] LF2N
			\end{itemize}\} \\ 
			Description: Reads the internal frequency modulation signal source. \\
			Command: SOUR:FM:INT:SOUR? \\
			\\

		\end{tabular}


		\begin{tabular}{N}
			\hline
			\bfseries ModFM-Sel \\ \hline
			\emph{Frequency Modulation Enable Selection} \\
			Data type: bool \{\begin{itemize}[noitemsep]
				\small
				\item[] OFF
				\item[] ON
			\end{itemize}\} \\
			Description: Activates/deactivates frequency modulation\\
			Command: SOUR:FM:STAT \emph{value} \\
			\\

		\end{tabular}


		\begin{tabular}{N}
			\hline
			\bfseries ModFM-Sts \\ \hline
			\emph{Frequency Modulation Enable Status} \\
			Data type: bool \{\begin{itemize}[noitemsep]
				\small
				\item[] OFF
				\item[] ON
			\end{itemize}\} \\
			Description: Gets frequency modulation state. \\
			Command: SOUR:FM:STAT? \\
			\\
			
		\end{tabular}


		\begin{tabular}{N}
			\hline
			\bfseries ModFMExtImpd-Sel \\ \hline
			\emph{Impedance for External FM Signal Selection} \\
			Data type: enum \\  
			Description: Sets the impedance for an externally applied modulation signal.\begin{itemize}[noitemsep]
				\small
				\item[] \textbf{HIGH}
                                \item[] Impedance bigger than 100.000 Ohm.
                                \item[] \textbf{G50}
                                \item[] Impedance equal to 50 Ohm.
			\end{itemize} \\
			Command: SOUR:INP:MOD:IMP \emph{value} \\
			\\

		\end{tabular}


		\begin{tabular}{N}
			\hline
			\bfseries ModFMExtImpd-Sts \\ \hline
			\emph{Impedance for External FM Signal Status} \\
			Data type: enum \{\begin{itemize}[noitemsep]
				\small
				\item[] HIGH
				\item[] G50
			\end{itemize}\} \\ 
			Description: Reads the impedance for the external frequency modulation signal. \\
			Command: SOUR:INP:MOD:IMP? \\
			\\

		\end{tabular}


		\begin{tabular}{N}
			\hline
			\bfseries ModFMDev-SP \\ \hline
			\emph{FM Deviation Set Point} \\
			Data type: float \\
			Unit: Hz \\  
			Range: 0 Hz to dynamic \\
			Description: Sets the deviation of the frequency modulation signals. The maximum deviation depends on the set RF frequency and the selected modulation mode. \\
			Command: SOUR:FM:DEV \emph{value} \\
			\\
			
		\end{tabular}


		\begin{tabular}{N}
			\hline
			\bfseries ModFMDev-RB \\ \hline
			\emph{FM Deviation Read Back} \\
			Data type: float \\
			Unit: Hz \\
			Description: Reads the deviation of the frequency modulation. \\
			Command: SOUR:FM:DEV? \\
			\\

		\end{tabular}


		\begin{tabular}{N}
			\hline
			\bfseries ModPulMPeriod-SP \\ \hline
			\emph{Pulse Modulation Period Set Point} \\
			Data type: float \\
			Unit: us \\ 
			Range: 20 ns to 100 s \\
			Description: Sets the period of the generated pulse. The period determines the repetition frequency of the internal signal. \\
			Command: SOUR:PULM:PER \emph{value} \\
			\\
			
		\end{tabular}


		\begin{tabular}{N}
			\hline
			\bfseries ModPulMPeriod-RB \\ \hline
			\emph{Pulse Modulation Period Read Back} \\
			Data type: float \\
			Unit: us \\
			Description: Reads the period of the generated pulse. \\
			Command: SOUR:PULM:PER? \\
			\\

		\end{tabular}


		\begin{tabular}{N}
			\hline
			\bfseries ModPulMWid-SP \\ \hline
			\emph{Pulse Width Set Point} \\
			Data type: float \\
			Unit: us \\ 
			Range: 5 ns to 100 s \\
			Description: Sets the width of the generated pulse. The width determines the pulse length. The pulse width must be at least 20ns less than the set pulse period. \\
			Command: SOUR:PULM:WIDTH \emph{value} \\
			\\
			
		\end{tabular}


		\begin{tabular}{N}
			\hline
			\bfseries ModPulMWid-RB \\ \hline
			\emph{Pulse Width Read Back} \\
			Data type: float \\
			Unit: us \\
			Description: Reads the width of the generated pulse. \\
			Command: SOUR:PULM:WIDTH? \\
			\\

		\end{tabular}


		\begin{tabular}{N}
			\hline
			\bfseries ModPulMMode-Sel \\ \hline
			\emph{Pulse Modulation Mode Selection} \\
			Data type: enum \\  
			Description: Sets the mode of the pulse generator.\begin{itemize}[noitemsep]
				\small
				\item[] \textbf{SINGle}
                                \item[] Sets the mode of the pulse generator.
                                \item[] \textbf{DOUBle}
                                \item[] Enables double pulse generation. The two pulses are generated in one pulse period.
				\item[] \textbf{PTRain}
                                \item[] A user-defined pulse train is generated The pulse train is defined by value pairs of on and off times that can be entered in a pulse train list.

			\end{itemize} \\
			Command: SOUR:PULM:MODE \emph{value} \\
			\\

		\end{tabular}


		\begin{tabular}{N}
			\hline
			\bfseries ModPulMMode-Sts \\ \hline
			\emph{Pulse Modulation Mode Status} \\
			Data type: enum \{\begin{itemize}[noitemsep]
				\small
				\item[] SING
				\item[] DOUB
				\item[] PTR
			\end{itemize}\} \\ 
			Description: Reads the pulse modulation mode. \\
			Command: SOUR:PULM:MODE? \\
			\\

		\end{tabular}


		\begin{tabular}{N}
			\hline
			\bfseries ModPulMPol-Sel \\ \hline
			\emph{Pulse Modulation Polarity Selection} \\
			Data type: enum \\  
			Description: Sets the polarity between modulating and modulated signal. This command is effective only for an external modulation signal.\begin{itemize}[noitemsep]
				\small
				\item[] \textbf{NORMal}
                                \item[] Impedance bigger than 100.000 Ohm.
                                \item[] \textbf{INVerted}
                                \item[] Impedance equal to 50 Ohm.
			\end{itemize} \\
			Command: SOUR:PULM:POL \emph{value} \\
			Configuration needed:\begin{itemize}[noitemsep]
				\small
				\item[] CONFIG FOR EXTERNAL MODULATION SIGNAL
			\end{itemize} \\

		\end{tabular}


		\begin{tabular}{N}
			\hline
			\bfseries ModPulMPol-Sts \\ \hline
			\emph{Pulse Modulation Polarity Status} \\
			Data type: enum \{\begin{itemize}[noitemsep]
				\small
				\item[] NORM
				\item[] INV
			\end{itemize}\} \\ 
			Description: Reads the polarity between modulating and modulated signal. \\
			Command: SOUR:PULM:POL? \\
			\\

		\end{tabular}


		\begin{tabular}{N}
			\hline
			\bfseries ModPulM-Sel \\ \hline
			\emph{Pulse Modulation Enable Selection} \\
			Data type: bool \{\begin{itemize}[noitemsep]
				\small
				\item[] OFF
				\item[] ON
			\end{itemize}\} \\
			Description: Activates/deactivates pulse modulation\\
			Command: SOUR:PULM:STAT \emph{value} \\
			\\

		\end{tabular}


		\begin{tabular}{N}
			\hline
			\bfseries ModPulM-Sts \\ \hline
			\emph{Pulse Modulation Enable Status} \\
			Data type: bool \{\begin{itemize}[noitemsep]
				\small
				\item[] OFF
				\item[] ON
			\end{itemize}\} \\
			Description: Gets pulse modulation state. \\
			Command: SOUR:PULM:STAT? \\
			\\
			
		\end{tabular}


		\begin{tabular}{N}
			\hline
			\bfseries ModPulMDelay-SP \\ \hline
			\emph{Pulse Modulation Delay Set Point} \\
			Data type: float \\
			Unit: us \\ 
			Range: 10 ns to 100 s \\
			Description: Sets the pulse delay. \\
			Command: SOUR:PULM:DEL \emph{value} \\
			\\
			
		\end{tabular}


		\begin{tabular}{N}
			\hline
			\bfseries ModPulMDelay-RB \\ \hline
			\emph{Pulse Modulation Delay Read Back} \\
			Data type: float \\
			Unit: us \\
			Description: Reads delay for the pulse modulation. \\
			Command: SOUR:PULM:DEL? \\
			\\

		\end{tabular}


		\begin{tabular}{N}
			\hline
			\bfseries ModPulMSrc-Sel \\ \hline
			\emph{Pulse Modulation Signal Source Selection} \\
			Data type: enum \\  
			Description: Selects the source for the pulse modulation signal.\begin{itemize}[noitemsep]
				\small
				\item[] \textbf{INTernal}
                                \item[] The internally generated rectangular signal is used for the pulse modulation.
                                \item[] \textbf{EXTernal}
                                \item[] The signal applied externally via the EXT MOD connector isused for the pulse modulation.
			\end{itemize} \\
			Command: SOUR:PULM:SOUR \emph{value} \\
			\\

		\end{tabular}


		\begin{tabular}{N}
			\hline
			\bfseries ModPulMSrc-Sts \\ \hline
			\emph{Pulse Modulation Signal Source Status} \\
			Data type: enum \{\begin{itemize}[noitemsep]
				\small
				\item[] INT
				\item[] EXT
			\end{itemize}\} \\ 
			Description: Reads the source for the pulse modulation signal. \\
			Command: SOUR:PULM:SOUR? \\
			\\

		\end{tabular}


		\begin{tabular}{N}
			\hline
			\bfseries ModPulMTrigMode-Sel \\ \hline
			\emph{Pulse Modulation Trigger Mode Selection} \\
			Data type: enum \\  
			Description: Selects the trigger mode for pulse modulation. \begin{itemize}[noitemsep]
				\small
				\item[] \textbf{AUTO}
                                \item[] The pulse modulation is generated continuously.
                                \item[] \textbf{EXTernal}
                                \item[] The pulse modulation is triggered by an external trigger event. The trigger signal is supplied via the PULSE EXT connector.
				\item[] \textbf{EGATe}
				\item[] The pulse modulation is gated by an external gate signal. The signal is supplied via the PULSE EXT connector.
			\end{itemize} \\
			Command: SOUR:PULM:TRIG:MODE \emph{value} \\
			\\

		\end{tabular}


		\begin{tabular}{N}
			\hline
			\bfseries ModPulMTrigMode-Sts \\ \hline
			\emph{Pulse Modulation Trigger Mode Status} \\
			Data type: enum \{\begin{itemize}[noitemsep]
				\small
				\item[] AUTO
				\item[] EXT
				\item[] EGAT
			\end{itemize}\} \\ 
			Description: Reads the pulse modulation trigger mode. \\
			Command: SOUR:PULM:TRIG:MODE? \\
			\\

		\end{tabular}


		\begin{tabular}{N}
			\hline
			\bfseries ModPulMExtGatePol-Sel \\ \hline
			\emph{Gate Polarity for the Pulse Modulation Selection} \\
			Data type: enum \\  
			Description: Selects the polarity of the Gate signal. The signal is supplied via the PULSE EXT connector. \begin{itemize}[noitemsep]
				\small
				\item[] \textbf{NORMal}
                                \item[] \textbf{INVerted}
			\end{itemize} \\
			Command: SOUR:PULM:TRIG:EXT:GATE:POL \emph{value} \\
			\\

		\end{tabular}


		\begin{tabular}{N}
			\hline
			\bfseries ModPulMExtGatePol-Sts \\ \hline
			\emph{Gate Polarity Pulse Modulation Status} \\
			Data type: enum \{\begin{itemize}[noitemsep]
				\small
				\item[] NORM
				\item[] INV
			\end{itemize}\} \\ 
			Description: Reads the polarity of the Gate signal for the pulse modulation. \\
			Command: SOUR:PULM:TRIG:EXT:GATE:POL? \\
			\\

		\end{tabular}


		\begin{tabular}{N}
			\hline
			\bfseries ModPulMExtImpdTrig-Sel \\ \hline
			\emph{Impedance for External Pulse Modulation Trigger Selection} \\
			Data type: enum \\  
			Description: Selects the impedance for external pulse trigger. \begin{itemize}[noitemsep]
				\small
				\item[] \textbf{G50}
				\item[] 50 Ohm impedance.
                                \item[] \textbf{G10K}
				\item[] 10 kOhm impedance.
			\end{itemize} \\
			Command: SOUR:PULM:TRIG:EXT:IMP \emph{value} \\
			\\

		\end{tabular}


		\begin{tabular}{N}
			\hline
			\bfseries ModPulMExtImpdTrig-Sts \\ \hline
			\emph{Impedance for External Pulse Modulation Trigger Status} \\
			Data type: enum \{\begin{itemize}[noitemsep]
				\small
				\item[] G50
				\item[] G10K
			\end{itemize}\} \\ 
			Description: Reads the impedance for the pulse trigger. \\
			Command: SOUR:PULM:TRIG:EXT:IMP? \\
			\\

		\end{tabular}


		\begin{tabular}{N}
			\hline
			\bfseries ModPulMExtSlopeTrig-Sel \\ \hline
			\emph{Slope Polarity of an External Trigger for Pulse Modulation Selection} \\
			Data type: enum \\  
			Description: Selects the polarity of the active slope of an applied trigger ate the PULSE EXT connector. \begin{itemize}[noitemsep]
				\small
				\item[] \textbf{NEGative}
                                \item[] \textbf{POSitive}
			\end{itemize} \\
			Command: SOUR:PULM:TRIG:EXT:SLOP \emph{value} \\
			\\

		\end{tabular}


		\begin{tabular}{N}
			\hline
			\bfseries ModPulMExtSlopeTrig-Sts \\ \hline
			\emph{Slope Polarity of an External Trigger for Pulse Modulation Status} \\
			Data type: enum \{\begin{itemize}[noitemsep]
				\small
				\item[] NEG
				\item[] POS
			\end{itemize}\} \\ 
			Description: Reads the slope polarity of an external pulse trigger. \\
			Command: SOUR:PULM:TRIG:EXT:SLOP? \\
			\\

		\end{tabular}


		\begin{tabular}{N}
			\hline
			\bfseries ModPG-Sel \\ \hline
			\emph{Pulse Generator Enable Selection} \\
			Data type: bool \{\begin{itemize}[noitemsep]
				\small
				\item[] OFF
				\item[] ON
			\end{itemize}\} \\
			Description: Activates/deactivates the output of the video/sync signal at the PULSE VIDEO connector at the rear of the instrument. The signal output and the pulse generator are automatically switched on with activation of pulse modulation if pulse generator is selected as modulation source. The signal output can be switched off subsequently. \\
			Command: SOUR:PGEN:STAT \emph{value} \\
			\\

		\end{tabular}


		\begin{tabular}{N}
			\hline
			\bfseries ModPG-Sts \\ \hline
			\emph{Pulse Generator Enable Status} \\
			Data type: bool \{\begin{itemize}[noitemsep]
				\small
				\item[] OFF
				\item[] ON
			\end{itemize}\} \\
			Description: Gets pulse generator state. \\
			Command: SOUR:PGEN:STAT? \\
			\\
			
		\end{tabular}


		\begin{tabular}{N}
			\hline
			\bfseries ModLFOut-Sel \\ \hline
			\emph{LF Output Enable Selection} \\
			Data type: bool \{\begin{itemize}[noitemsep]
				\small
				\item[] OFF
				\item[] ON
			\end{itemize}\} \\
			Description: Activates/deactivates the LF output. \\
			Command: SOUR:LFO:STAT \emph{value} \\
			\\

		\end{tabular}


		\begin{tabular}{N}
			\hline
			\bfseries ModLFOut-Sts \\ \hline
			\emph{LF Output Enable Status} \\
			Data type: bool \{\begin{itemize}[noitemsep]
				\small
				\item[] OFF
				\item[] ON
			\end{itemize}\} \\
			Description: Gets the LF output state. \\
			Command: SOUR:LFO:STAT? \\
			\\
			
		\end{tabular}


		\begin{tabular}{N}
			\hline
			\bfseries ModLFOutSrc-Sel \\ \hline
			\emph{Internal LF Source Selection} \\
			Data type: enum \\  
			Description: Selects the internal source to be used for the LF Output signal. The available selection depends on the options fitted. If test signals for avionic systems are generated, the sources are preset and cannot be changed. \begin{itemize}[noitemsep]
				\small
                                \item[] \textbf{LF1}
                                \item[] Internal LF generator 1.
                                \item[] \textbf{LF2}
                                \item[] Internal LF generator 2.
                                \item[] \textbf{LF12}
                                \item[] Selects both internal generators. LF freque     ncy and modulation depth can be set separately. The added modulation depths of the two modulation generators must not exceed the overall modulation depth. This selection enables two-tone AM modulation.
                                \item[] \textbf{NOISe}
                                \item[] Selects noise signal. The modulation signal      is white noise either with Gaussian distribution or equal distribution. This setting affects all analog modulations which use the noise generator as the internal modulation source.
                                \item[] \textbf{LF1Nois}
                                \item[] Internal LF generator 1 and the noise signal. In addition to the AM modulation signal, white noise is used as a modulation signal.
                                \item[] \textbf{LF2Nois}
                                \item[] Internal LF generator 2 and the noise signal. In addition to the AM modulation signal, white noise is used as a modulation signal.
			\end{itemize} \\
			Command: SOUR:LFO:SOUR \emph{value} \\
			\\

		\end{tabular}


		\begin{tabular}{N}
			\hline
			\bfseries ModLFOutSrc-Sts \\ \hline
			\emph{Internal LF Source Status} \\
			Data type: enum \{\begin{itemize}[noitemsep]
				\small
				\item[] LF1
				\item[] LF2
				\item[] LF12
				\item[] NOIS
				\item[] LF1N
				\item[] LF2N

			\end{itemize}\} \\ 
			Description: Reads internal LF source. \\
			Command: SOUR:LFO:SOUR? \\
			\\

		\end{tabular}



	% TABLE: Trigger Functionalities
	\subsection{Trigger Functionalities}\label{pvgroup:function} %LABEL NOT CHANGED YET

		\paragraph{} % This paragraph aligns the first tabular with the others


		\begin{tabular}{N}
			\hline
			\bfseries TrigInpSlopePol-Sel \\ \hline
			\emph{Input Trigger Polarity Selection} \\
			Data type: enum \\   
			Description: Selects the polarity of the active slope of an externally applied trigger signal at the trigger input (BNC connector at the rear of the instrument). The setting is effective for both inputs at the same time.\begin{itemize}[noitemsep]
				\small
				\item[] \textbf{NEGative}
                                \item[] \textbf{POSitive}
			\end{itemize} \\
			Command: SOUR:INP:TRIG:SLOP \emph{value} \\
			\\

		\end{tabular}


		\begin{tabular}{N}
			\hline
			\bfseries ModPulMExtSlopeTrig-Sts \\ \hline
			\emph{Input Trigger Polarity Status} \\
			Data type: enum \{\begin{itemize}[noitemsep]
				\small
				\item[] NEG
				\item[] POS
			\end{itemize}\} \\ 
			Description: Reads the input trigger polarity. \\
			Command: SOUR:INP:TRIG:SLOP? \\
			\\

		\end{tabular}


		\begin{tabular}{N}
			\hline
			\bfseries TrigFSweepSrc-Sel \\ \hline
			\emph{Frequency Sweep Trigger Source Selection} \\
			Data type: enum \\   
			Description: Selects the trigger source for the RF frequency sweep.\begin{itemize}[noitemsep]
				\small
				\item[] \textbf{AUTO}
				\item[] The trigger is free-running, i.e. the trigger condition is fulfilled continuously. As soon as one sweep is finished, the next sweep is started.
                                \item[] \textbf{SINGle}
				\item[] One complete sweep cycle is triggered by the command "Execute Single Sweep".
				\item[] \textbf{EXTernal}
				\item[] The sweep is triggered externally via the INST TRIG connector.
                                \item[] \textbf{EAUTo}
				\item[] The sweep is triggered externally via the INST TRIG connector. As soon as one sweep is finished, the next sweep is started. A second trigger event stops the sweep at the current frequency, a third trigger event starts the trigger at the start frequency, and so on.

			\end{itemize} \\
			Command: TRIG:FSW:SOUR \emph{value} \\
			\\

		\end{tabular}


		\begin{tabular}{N}
			\hline
			\bfseries TrigFSweepSrc-Sts \\ \hline
			\emph{Frequency Sweep Trigger Source Status} \\
			Data type: enum \{\begin{itemize}[noitemsep]
				\small
				\item[] AUTO
				\item[] SING
				\item[] EXT
				\item[] EAUT
			\end{itemize}\} \\ 
			Description: Reads the trigger source for the frequency sweep. \\
			Command: TRIG:FSW:SOUR? \\
			\\

		\end{tabular}


		\begin{tabular}{N}
			\hline
			\bfseries TrigPSweepSrc-Sel \\ \hline
			\emph{RF Level Sweep Trigger Source Selection} \\
			Data type: enum \\   
			Description: Selects the trigger source for the RF level sweep.\begin{itemize}[noitemsep]
				\small
				\item[] \textbf{AUTO}
				\item[] The trigger is free-running, i.e. the trigger condition is fulfilled continuously. As soon as one sweep is finished, the next sweep is started.
                                \item[] \textbf{SINGle}
				\item[] One complete sweep cycle is triggered by the command "Execute Single Sweep".
				\item[] \textbf{EXTernal}
				\item[] The sweep is triggered externally via the INST TRIG connector.
                                \item[] \textbf{EAUTo}
				\item[] The sweep is triggered externally via the INST TRIG connector. As soon as one sweep is finished, the next sweep is started. A second trigger event stops the sweep at the current frequency, a third trigger event starts the trigger at the start frequency, and so on.

			\end{itemize} \\
			Command: TRIG:PSW:SOUR \emph{value} \\
			\\

		\end{tabular}


		\begin{tabular}{N}
			\hline
			\bfseries TrigPSweepSrc-Sts \\ \hline
			\emph{RF Level Sweep Trigger Source Status} \\
			Data type: enum \{\begin{itemize}[noitemsep]
				\small
				\item[] AUTO
				\item[] SING
				\item[] EXT
				\item[] EAUT
			\end{itemize}\} \\ 
			Description: Reads the trigger source for the RF level sweep. \\
			Command: TRIG:PSW:SOUR? \\
			\\

		\end{tabular}


		\begin{tabular}{N}
			\hline
			\bfseries TrigLFSweepSrc-Sel \\ \hline
			\emph{LF Sweep Trigger Source Selection} \\
			Data type: enum \\   
			Description: Selects the trigger source for the LF sweep.\begin{itemize}[noitemsep]
				\small
				\item[] \textbf{AUTO}
				\item[] The trigger is free-running, i.e. the trigger condition is fulfilled continuously. As soon as one sweep is finished, the next sweep is started.
                                \item[] \textbf{SINGle}
				\item[] One complete sweep cycle is triggered by the command "Execute Single Sweep".
				\item[] \textbf{EXTernal}
				\item[] The sweep is triggered externally via the INST TRIG connector.
                                \item[] \textbf{EAUTo}
				\item[] The sweep is triggered externally via the INST TRIG connector. As soon as one sweep is finished, the next sweep is started. A second trigger event stops the sweep at the current frequency, a third trigger event starts the trigger at the start frequency, and so on.

			\end{itemize} \\
			Command: TRIG:LFFS:SOUR \emph{value} \\
			\\

		\end{tabular}


		\begin{tabular}{N}
			\hline
			\bfseries TrigLFSweepSrc-Sts \\ \hline
			\emph{LF Sweep Trigger Source Status} \\
			Data type: enum \{\begin{itemize}[noitemsep]
				\small
				\item[] AUTO
				\item[] SING
				\item[] EXT
				\item[] EAUT
			\end{itemize}\} \\ 
			Description: Reads the trigger source for the LF sweep. \\
			Command: TRIG:LFFS:SOUR? \\
			\\

		\end{tabular}


		\begin{tabular}{N}
			\hline
			\bfseries TrigAllSweepSrc-Sel \\ \hline
			\emph{All Sweeps Trigger Source Selection} \\
			Data type: enum \\   
			Description: Selects the trigger source for all the sweeps.\begin{itemize}[noitemsep]
				\small
				\item[] \textbf{AUTO}
				\item[] The trigger is free-running, i.e. the trigger condition is fulfilled continuously. As soon as one sweep is finished, the next sweep is started.
                                \item[] \textbf{SINGle}
				\item[] One complete sweep cycle is triggered by the command "Execute Single Sweep".
				\item[] \textbf{EXTernal}
				\item[] The sweep is triggered externally via the INST TRIG connector.
                                \item[] \textbf{EAUTo}
				\item[] The sweep is triggered externally via the INST TRIG connector. As soon as one sweep is finished, the next sweep is started. A second trigger event stops the sweep at the current frequency, a third trigger event starts the trigger at the start frequency, and so on.

			\end{itemize} \\
			Command: TRIG:SWE:SOUR \emph{value} \\
			\\

		\end{tabular}


		\begin{tabular}{N}
			\hline
			\bfseries TrigAllSweep-Cmd \\ \hline
			\emph{Trigger All Sweeps Command} \\
			Data type: bool \{\begin{itemize}[noitemsep]
				\small
				\item[] OFF
				\item[] ON
			\end{itemize}\} \\
			Description: Starts all sweeps which are activated for the respective path. \\
			Command: TRIG:SWE:IMM \emph{value} \\
			\\

		\end{tabular}


		\begin{tabular}{N}
			\hline
			\bfseries TrigSweep-Cmd \\ \hline
			\emph{Trigger Activated Sweep Command} \\
			Data type: bool \{\begin{itemize}[noitemsep]
				\small
				\item[] OFF
				\item[] ON
			\end{itemize}\} \\
			Description: The command immediately starts the activated sweep. The command performs a single sweep and therefore applies to sweep mode AUTO with sweep source SINGle.  \\
			Command: TRIG:IMM \emph{value} \\
			\\

		\end{tabular}


		\begin{tabular}{N}
			\hline
			\bfseries TrigFSweep-Cmd \\ \hline
			\emph{Execute Single Frequency Sweep Command} \\
			Data type: bool \{\begin{itemize}[noitemsep]
				\small
				\item[] OFF
				\item[] ON
			\end{itemize}\} \\
			Description: Immediately starts an RF frequency sweep cycle. The command is only effective for sweep mode "Single". \\
			Command: TRIG:FSW:IMM \emph{value} \\
			\\

		\end{tabular}


		\begin{tabular}{N}
			\hline
			\bfseries TrigPSweep-Cmd \\ \hline
			\emph{Start RF Level Sweep Command} \\
			Data type: bool \{\begin{itemize}[noitemsep]
				\small
				\item[] OFF
				\item[] ON
			\end{itemize}\} \\
			Description: Immediately starts an RF level sweep. The command is only effective for sweep mode "Single". \\
			Command: TRIG:PSW:IMM \emph{value} \\
			\\

		\end{tabular}


	% TABLE: Reference Oscillator Functionalities
	\subsection{Reference Oscillator Functionalities}\label{pvgroup:function} %LABEL NOT CHANGED YET

		\paragraph{} % This paragraph aligns the first tabular with the others


		\begin{tabular}{N}
			\hline
			\bfseries RoscExtBwid-Sel \\ \hline
			\emph{Synchronization Bandwidth Mode Selection} \\
			Data type: enum \\   
			Description: Sets the synchronization bandwidth for an external reference signal.\begin{itemize}[noitemsep]
				\small
				\item[] \textbf{WIDE}
				\item[] The internal 10-MHz OCXO is synchronized to the external signal. This setting is recommended if the phase noise of the external signal is worse than the phase noise of the internal OCXO.
                                \item[] \textbf{NARRow}
				\item[] This mode is recommended for precise reference sources of high spectral purity. The internal 10-MHz OCXO is bypassed and the external signal synchronizes a 100-MHz reference oscillator directly.
			\end{itemize} \\
			Command: SOUR:ROSC:EXT:SBAN \emph{value} \\
			\\

		\end{tabular}


		\begin{tabular}{N}
			\hline
			\bfseries RoscExtBwid-Sts \\ \hline
			\emph{Get Synchronization Bandwidth Mode} \\
			Data type: enum \{\begin{itemize}[noitemsep]
				\small
				\item[] WIDE
				\item[] NARR
			\end{itemize}\} \\ 
			Description: Reads the synchronization bandwidth for an external reference signal. \\
			Command: SOUR:ROSC:EXT:SBAN? \\
			\\

		\end{tabular}


		\begin{tabular}{N}
			\hline
			\bfseries RoscSrc-Sel \\ \hline
			\emph{Reference Frequency Source Selection} \\
			Data type: enum \\   
			Description: Selects the reference frequency source.\begin{itemize}[noitemsep]
				\small
				\item[] \textbf{INTernal}
				\item[] The internal reference oscillator is used.
                                \item[] \textbf{EXTernal}
				\item[] An external reference signal is used. It must be input at the REF IN connector at the rear of the instrument.
			\end{itemize} \\
			Command: SOUR:ROSC:SOUR \emph{value} \\
			\\

		\end{tabular}


		\begin{tabular}{N}
			\hline
			\bfseries RoscSrc-Sts \\ \hline
			\emph{Synchronization Bandwidth Mode Status} \\
			Data type: enum \{\begin{itemize}[noitemsep]
				\small
				\item[] WIDE
				\item[] NARR
			\end{itemize}\} \\ 
			Description: Reads the reference frequency source. \\
			Command: SOUR:ROSC:SOUR? \\
			\\

		\end{tabular}


		\begin{tabular}{N}
			\hline
			\bfseries RoscExtFreq-Sel \\ \hline
			\emph{External Reference Frequency Selection} \\
			Data type: enum \\   
			Description: Selects the external reference frequency.\begin{itemize}[noitemsep]
				\small
				\item[] \textbf{5MHZ}
                                \item[] \textbf{10MHZ}
				\item[] \textbf{13MHZ}

			\end{itemize} \\
			Command: SOUR:ROSC:EXT:FREQ \emph{value} \\
			\\

		\end{tabular}


		\begin{tabular}{N}
			\hline
			\bfseries RoscExtFreq-Sts \\ \hline
			\emph{External Reference Frequency Status} \\
			Data type: enum \{\begin{itemize}[noitemsep]
				\small
				\item[] 5MHZ
				\item[] 10MHZ
				\item[] 13MHZ
			\end{itemize}\} \\ 
			Description: Reads the reference external reference frequency. \\
			Command: SOUR:ROSC:EXT:FREQ? \\
			\\

		\end{tabular}


		\begin{tabular}{N}
			\hline
			\bfseries RoscExt-Sel \\ \hline
			\emph{RF Output Switch Off Enable Selection} \\
			Data type: bool \{\begin{itemize}[noitemsep]
				\small
				\item[] OFF
				\item[] ON
			\end{itemize}\} \\
			Description: Selects if RF output is automatically switched off, when in external source mode no reference signal is supplied. This setting ensures that no improper RF signal due to the missing external reference signal is output and used for measurements. \\
			Command: SOUR:ROSC:EXT:RFOF:STAT \emph{value} \\
			\\

		\end{tabular}


		\begin{tabular}{N}
			\hline
			\bfseries RoscExt-Sts \\ \hline
			\emph{RF Output Switch Off Enable Status} \\
			Data type: bool \{\begin{itemize}[noitemsep]
				\small
				\item[] OFF
				\item[] ON
			\end{itemize}\} \\
			Description: Gets if the RF output switch off state. \\
			Command: SOUR:ROSC:EXT:RFOF:STAT? \\
			\\
			
		\end{tabular}


	% TABLE: Clock Synthesis Functionalities
	\subsection{Clock Synthesis Functionalities}\label{pvgroup:function} %LABEL NOT CHANGED YET

		\paragraph{} % This paragraph aligns the first tabular with the others

		\begin{tabular}{N}
			\hline
			\bfseries CsynFreq-SP \\ \hline
			\emph{Clock Synthesis Frequency Set Point} \\
			Data type: float \\
			Unit: Hz \\
			Range: 100 kHz to 1.5 GHz \\
			Description: Sets the frequency of the clock synthesis output signal. \\
			Command: CSYN:FREQ \emph{value} \\
			\\

		\end{tabular}


		\begin{tabular}{N}
			\hline
			\bfseries CsynFreq-RB \\ \hline
			\emph{Clock Synthesis Frequency Read Back} \\
			Data type: float \\
			Unit: Hz \\
			Description: Reads the frequency of the clock synthesis signal. \\
			Command: CSYN:FREQ? \\
			\\

		\end{tabular}


		\begin{tabular}{N}
			\hline
			\bfseries Csyn-Sel \\ \hline
			\emph{Clock Synthesis Enable Selection} \\
			Data type: bool \{\begin{itemize}[noitemsep]
				\small
				\item[] OFF
				\item[] ON
			\end{itemize}\} \\
			Description: Activates/deactivates generation of a system clock for differential outputs CLK SYN and CLK SYN N at the rear of the instrument. \\
			Command: CSYN:STAT \emph{value} \\
			\\

		\end{tabular}


		\begin{tabular}{N}
			\hline
			\bfseries Csyn-Sts \\ \hline
			\emph{Clock Synthesis Enable Status} \\
			Data type: bool \{\begin{itemize}[noitemsep]
				\small
				\item[] OFF
				\item[] ON
			\end{itemize}\} \\
			Description: Gets clock synthesis state. \\
			Command: CSYN:STAT? \\
			\\
			
		\end{tabular}
%

		\begin{tabular}{N}
			\hline
			\bfseries CsynOffset-SP \\ \hline
			\emph{Clock Synthesis DC Offset Set Point} \\
			Data type: float \\
			Unit: V \\
			Range: -5 V to 5 V \\
			Description: Sets a DC offset which is added to both clock synthesis output signals. \\
			Command: CSYN:OFFS \emph{value} \\
			\\

		\end{tabular}


		\begin{tabular}{N}
			\hline
			\bfseries CsynOffset-RB \\ \hline
			\emph{Clock Synthesis DC Offset Read Back} \\
			Data type: float \\
			Unit: V \\
			Description: Reads the DC offset for the clock synthesis signal value. \\
			Command: CSYN:OFFS? \\
			\\

		\end{tabular}


		\begin{tabular}{N}
			\hline
			\bfseries CsynOffset-Sel \\ \hline
			\emph{Clock Synthesis DC Offset Enable Selection} \\
			Data type: bool \{\begin{itemize}[noitemsep]
				\small
				\item[] OFF
				\item[] ON
			\end{itemize}\} \\
			Description: Activates the addition of the DC offset to both clock synthesis output signals. \\
			Command: CSYN:OFFS:STAT \emph{value} \\
			\\

		\end{tabular}


		\begin{tabular}{N}
			\hline
			\bfseries CsynOffset-Sts \\ \hline
			\emph{Clock Synthesis DC Offset Enable Status} \\
			Data type: bool \{\begin{itemize}[noitemsep]
				\small
				\item[] OFF
				\item[] ON
			\end{itemize}\} \\
			Description: Gets clock synthesis DC Offset status. \\
			Command: CSYN:OFFS:STAT? \\
			\\
			
		\end{tabular}


	% TABLE: Clock Synthesis Functionalities
	\subsection{Clock Synthesis Functionalities}\label{pvgroup:function} %LABEL NOT CHANGED YET

		\paragraph{} % This paragraph aligns the first tabular with the others

		\begin{tabular}{N}
			\hline
			\bfseries NoisBwid-SP \\ \hline
			\emph{Noise Bandwidth Set Point} \\
			Data type: float \\
			Unit: Hz \\
			Range: 100 kHz to 10 MHz \\
			Description: Sets the noise level in the system bandwidth for enabled bandwidth limitation. \\
			Command: SOUR:NOIS:BWID \emph{value} \\
			\\

		\end{tabular}


		\begin{tabular}{N}
			\hline
			\bfseries NoisBwid-RB \\ \hline
			\emph{Noise Bandwidth Read Back} \\
			Data type: float \\
			Unit: Hz \\
			Description: Reads noise bandwidth limitation. \\
			Command: SOUR:NOIS:BWID? \\
			\\

		\end{tabular}


		\begin{tabular}{N}
			\hline
			\bfseries NoisBwid-Sel \\ \hline
			\emph{Noise Bandwidth Limitation Enable Selection} \\
			Data type: bool \{\begin{itemize}[noitemsep]
				\small
				\item[] OFF
				\item[] ON
			\end{itemize}\} \\
			Description: Enables /disables bandwidth limitation of noise. \\
			Command: SOUR:NOIS:BWID:STAT \emph{value} \\
			\\

		\end{tabular}


		\begin{tabular}{N}
			\hline
			\bfseries NoisBwid-Sts \\ \hline
			\emph{Noise Bandwidth Limitation Enable Status} \\
			Data type: bool \{\begin{itemize}[noitemsep]
				\small
				\item[] OFF
				\item[] ON
			\end{itemize}\} \\
			Description: Gets bandwidth limitation of noise status. \\
			Command: SOUR:NOIS:BWID:STAT? \\
			\\
			
		\end{tabular}


		\begin{tabular}{N}
			\hline
			\bfseries NoisDist-Sel \\ \hline
			\emph{Noise Distribution Selection} \\
			Data type: enum \\   
			Description: Selects the noise power density distribution.\begin{itemize}[noitemsep]
				\small
				\item[] \textbf{GAUSs}
                                \item[] \textbf{EQUal}

			\end{itemize} \\
			Command: SOUR:NOIS:DIST \emph{value} \\
			\\

		\end{tabular}


		\begin{tabular}{N}
			\hline
			\bfseries NoisDist-Sts \\ \hline
			\emph{Noise Distribution Status} \\
			Data type: enum \{\begin{itemize}[noitemsep]
				\small
				\item[] GAUS
				\item[] EQU

			\end{itemize}\} \\ 
			Description: Reads the noise power density distribution \\
			Command: SOUR:NOIS:DIST? \\
			\\

		\end{tabular}


		\begin{tabular}{N}
			\hline
			\bfseries NoisRelatLvl-Mon \\ \hline
			\emph{Noise Relative Level} \\
			Data type: float \\
			Unit: Hz \\
			Description: Monitors the level of the noise signal pr Hz in the total bandwidth. \\
			Command: SOUR:NOIS:LEV:REL? \\
			\\

		\end{tabular}


		\begin{tabular}{N}
			\hline
			\bfseries NoisAbsLvl-Mon \\ \hline
			\emph{Noise Absolute Level} \\
			Data type: float \\
			Unit: Hz \\
			Description: Monitors the noise signal in the system bandwidth for enabled bandwidth limitation. \\
			Command: SOUR:NOIS:LEV:ABS? \\
			\\

		\end{tabular}




---------
\end{document}
\grid
