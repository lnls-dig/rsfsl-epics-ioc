\documentclass[openany]{article}
\usepackage[a4paper,margin=1in,bottom=1.5in]{geometry} % define margins. Bottom margin is used to lift a little bit the page number.
\usepackage[english]{babel} % document language is english
\usepackage{tikz} % for drawing (currently not used).
\usepackage{graphicx} % for including images
\usepackage[export]{adjustbox}
\usepackage{fancyhdr} % used for creating headers and footers. only used in title page in this document.
\usepackage{tabularx} % creation of more complex tables
\usepackage{longtable} % tables can span multiple pages
\usepackage{array} % allow elements of tabular environment to have vertical alignment, e.g., center alignment.
\usepackage{nameref} % make it possible to reference by name
\usepackage{hyperref} % allow hiperlinks (links to other document parts and extern links)
\usepackage{etoc} % used for generation of section local table of contents
\usepackage{placeins}
\usepackage{siunitx} % SI units package
\usepackage{enumitem} % allows removing space between list items
\usepackage{xcolor,colortbl} % makes it possible to change table lines color

% Define graphics path
\graphicspath{{figs/}}

% Configure the cross reference hyper links color
\hypersetup{
    colorlinks=true,
    linkcolor=blue,
}

\renewcommand{\arraystretch}{2} % increase height of table rows
\newcolumntype{N}{p{14cm}} % new column type

\newcolumntype{C}{>{\centering\arraybackslash}X} % new column type for tabularx
						 % centered (\centering), adjust width in order to fill table width (X type)

% Configure header in 'titlepage'
%\pagestyle{fancy}
%\lhead{\includegraphics[width=4.5cm]{logo_cnpem}}
%\rhead{\includegraphics[width=4cm]{logo_lnls}}
%\renewcommand{\headrulewidth}{0pt}
%\setlength{\headheight}{52pt}
% Clean footer
%\fancyfoot{}

% increase table height factor a little bit (taller cells)
%\renewcommand{\arraystretch}{1.5}

%==== Begin DOCUMENT ====
\begin{document}

%--- Begin title page ---
\begin{titlepage}

% Add header to this page
%\thispagestyle{fancy}

% Center elements
\begin{center}

% title of title page
\topskip0pt % perfectly centered
\vspace*{\fill}
\textbf{\Huge RSFSL EPICS IOC User Guide}\\[20pt]
\textbf{\Huge Version 1.0}\\[20pt]
\textbf{\Huge October/2018}
\vspace*{\fill}

% footer of title page
\vfill
\textbf{Beam Diagnostics Group (DIG)}\\[5pt]
\textbf{Brazilian Synchrotron Light Laboratory (LNLS)}\\[5pt]
\textbf{Brazilian Center for Research in Energy and Materials (CNPEM)}
\end{center}

\end{titlepage}
%--- End of title page ---

\newpage
\pagestyle{plain} % restore default page style

%--- Table of contents ---
\tableofcontents

\newpage
%--- Section: RSSMX100A IOC ---
\section{RSSMX IOC}

	\paragraph{} The RSFSL IOC provides most of the FSL configuration parameters as EPICS PVs. Its goal is to facilitate the process of setting the desired values and building application-specific IOCs which could make use of a FSL IOC.

%--- Section: Document Overview ---
\section{Document Overview}

	\paragraph{} This document lists the IOC PVs along with their data type, limits, units, description, and related command. In most cases, a PV is a direct mapping of a FSL parameter, and its description is the same provided for the parameter in the Manufacturer Reference Manual. The Reference Manual provides all the information about the signal generator features and options. After a function or parameter is well understood, it should be easy to locate the associated PVs on this document.

%--- Section: IOC Configuration Steps ---
\section{IOC Configuration Steps}

	% Dependencies
	\paragraph{} This IOC requires \emph{EPICS base 3.14.12.5} and \emph{synApps 5.8}.

	% Edit Release File
	\paragraph{} When setting up the IOC, it is necessary to edit the \emph{RELEASE} file in the \emph{configure} directory to provide the right path to support modules. Edit the following lines:

	\begin{itemize}
		\item[] SUPPORT=/\textless path\textgreater/\textless to\textgreater/\textless synApps\textgreater
		\item[] EPICS\_BASE=/\textless path\textgreater/\textless to\textgreater/\textless epics\textgreater/\textless base\textgreater
		\item[] ASYN=\$(SUPPORT)/\textless path to asyn\textgreater
		\item[] STREAM=\$(SUPPORT)/\textless path to stream device\textgreater
		\item[] CALC=\$(SUPPORT)/\textless path to calc module\textgreater
		\item[] AUTOSAVE=\$(SUPPORT)/\textless path to autosave\textgreater
	\end{itemize}

	% Edit st.cmd file
	\paragraph{} Edit the \emph{st.cmd} file to set the FSL network address using the \emph{drvAsynIPPortConfigure} command. Load the \emph{rsfsl.db} with the \emph{dbLoadRecords} command and set the desired prefix for the records names. The records' names prefixes have two parts: \emph{P} and \emph{R}.

	\begin{itemize}
		\item[] drvAsynIPPortConfigure("\textless port name\textgreater", "\textless RSFSL IP{\textgreater} TCP",0,0,0)
		\item[] dbLoadRecords("\${TOP}/db/rsfsl.db", "P=\textless first prefix part\textgreater, R=\textless second prefix part\textgreater, PORT=\textless port name\textgreater")
	\end{itemize}

%--- Section: PVs Suffixes ---
\section{PVs Suffixes}

	\paragraph{} The records in this IOC fall into different categories depending on their functionalities. The categories are defined by the suffixes, according to Table 1.

	\begin{table}[!h]
		\center
		\caption{PVs Suffixes}
		\begin{tabular}{m{3cm} m{3cm} m{7cm}}
			\hline
			\bfseries Suffix & \bfseries Name & \bfseries Description \\ \hline
			-Sel & Selection & Selection functionalities \\ \hline
			-Sts & Status & Status monitoring functionalities \\ \hline
			-SP & Setpoint & Setting functionalities \\ \hline
			-RB & Readback & Readback \\ \hline
			-Mon & Monitor & Monitoring functionalities \\ \hline
			-Cmd & Command & Command functionalities\\ \hline

		\end{tabular}
	\end{table}

%--- Section: PV List ---
\section{PV List}

		\paragraph{} Each PV information block starts with the PV Channel Access name in bold text, followed by a longer, more descriptive name. The PV input data type comes next, followed by the limits or options, when relevant. A description of the PV function is then provided. When available, the associated spectrum analizer command (SCPI command) is listed. For PVs that only work when certain measure functions are selected, a table indicates the set of allowed functions at the end of the block.

		\newcommand{\FuncTableBorderColor}{gray!50} % define function table border color
		\newcommand{\nofunc}{\cellcolor{gray!20}\color{gray}} % define function table "not allowed function" cell color
		\newcommand{\yesfunc}{\cellcolor{white}\color{black}} % define function table "allowed function" cell color

		\bigskip
		\begin{tabular}{N}
			\hline
			\bfseries PVExample \\ \hline
			\emph{PV Name Example} \\
			Data type: The PV data type. \\
			Unit: show the datas unit, when applicable. \\
			Range: minimun and maximun permitted values, when applicable. \\
			Description: Description of the PV function. \\
			SCPI Command: Shows the related equipment remote command. \\
			Configuration: what configuration the equipment must have for this PV to be editable. \\

		\end{tabular}


	% General Functionalities: PVs with -Sel,-Sts,-SP,-RB suffixes
	\subsection{General PVs}\label{pvgroup:function}
		

		\paragraph{} % This paragraph aligns the first tabular with the others

		\begin{tabular}{N}
			\hline
			\bfseries RefLvl-RB \\ \hline
			\emph{Reference Level Read Back} \\
			Data type: float \\
			Unit: dBm \\
			Description: Read the reference level. \\
			SCPI Command: DISPlay[:WINDow]:TRACe:Y[:SCALe]:RLEVel? \\
			\\

		\end{tabular}

		\begin{tabular}{N}
			\hline
			\bfseries RefLvl-SP \\ \hline
			\emph{Reference Level Set Point} \\
			Data type: float \\
			Unit: dBm \\
			Description: Set the reference level. \\
			SCPI Command: DISPlay[:WINDow]:TRACe:Y[:SCALe]:RLEVel \emph{value} \\

		\end{tabular}

		\begin{tabular}{N}
			\hline
			\bfseries RefOff-RB \\ \hline
			\emph{Reference Level Offset Read Back} \\
			Data type: float \\
			Unit: dB \\
			Description: Get the reference level offset. \\
			SCPI Command: DISPlay[:WINDow]:TRACe:Y[:SCALe]:RLEVel:OFFSet? \\
			\\

		\end{tabular}

		\begin{tabular}{N}
			\hline
			\bfseries RefLvl-SP \\ \hline
			\emph{Reference Level Offset Set Point} \\
			Data type: float \\
			Unit: dB \\
			Description: Set the reference level offset. \\
			SCPI Command: DISPlay[:WINDow]:TRACe:Y[:SCALe]:RLEVel:OFFSet \emph{value} \\

		\end{tabular}

		\begin{tabular}{N}
			\hline
			\bfseries EnblContSwe-Sts \\ \hline
			\emph{Enable Continuous Sweep Status} \\
			Data type: bool \{\begin{itemize}[noitemsep]
				\small
				\item[] OFF
				\item[] ONN
			\end{itemize}\} \\
			Description: Get continuous sweep state. \\
			SCPI Command: INITiate:CONTinuous? \\
			\\

		\end{tabular}

		\begin{tabular}{N}
			\hline
			\bfseries EnblContSwe-Sel \\ \hline
			\emph{Enable Continuous Sweep Selection} \\
			Data type: bool \{\begin{itemize}[noitemsep]
				\small
				\item[] OFF
				\item[] ON
			\end{itemize}\} \\
			Description: Enable/Disable continuous sweep state. \\
			SCPI Command: INITiate:CONTinuous \emph{state} \\
			\\

		\end{tabular}

		\begin{tabular}{N}
			\hline
			\bfseries SwePts-RB \\ \hline
			\emph{Sweep Points Read Back} \\
			Data type: integer \\
			Description: Get the number of sweep points. \\
			SCPI Command: [SENSe:]SWEep:POINts? \\
			\\

		\end{tabular}

		\begin{tabular}{N}
			\hline
			\bfseries SwePts-SP \\ \hline
			\emph{Sweep Points Set Point} \\
			Data type: integer \\
			Description: Set the number of sweep points. \\
			SCPI Command: [SENSe:]SWEep:POINts \emph{value} \\

		\end{tabular}

		\begin{tabular}{N}
			\hline
			\bfseries SweTime-RB \\ \hline
			\emph{Sweep Time Read Back} \\
			Data type: float \\
			Unit: s \\
			Description: Get the sweep time. \\
			SCPI Command: [SENSe:]SWEep:TIME? \\
			\\

		\end{tabular}

		\begin{tabular}{N}
			\hline
			\bfseries SweTime-SP \\ \hline
			\emph{Sweep Time Set Point} \\
			Data type: float \\
			Unit: s \\
			Description: Set the sweep time. \\
			SCPI Command: [SENSe:]SWEep:TIME \emph{value} \\

		\end{tabular}

		\begin{tabular}{N}
			\hline
			\bfseries EnblSweTimeAuto-Sts \\ \hline
			\emph{Enable Automatic Sweep Time Status} \\
			Data type: bool \{\begin{itemize}[noitemsep]
				\small
				\item[] OFF
				\item[] ONN
			\end{itemize}\} \\
			Description: Get the state of the automatic coupling of the sweep time to the frequency span and bandwidth settings. \\
			SCPI Command: [SENSe:]SWEep:TIME:AUTO? \\
			\\

		\end{tabular}

		\begin{tabular}{N}
			\hline
			\bfseries EnblSweTimeAuto-Sel \\ \hline
			\emph{Enable Continuous Sweep Selection} \\
			Data type: bool \{\begin{itemize}[noitemsep]
				\small
				\item[] OFF
				\item[] ON
			\end{itemize}\} \\
			Description: Enable/Disable automatic coupling of the sweep time to the frequency span and bandwidth settings. \\
			SCPI Command: [SENSe:]SWEep:TIME:AUTO \emph{state} \\
			\\

		\end{tabular}

		\begin{tabular}{N}
			\hline
			\bfseries TraceMode-Sts \\ \hline
			\emph{Trace Mode Status} \\
			Data type: enum \{\begin{itemize}[noitemsep]
				\small
				\item[] \textbf{WRIT}
				\item[] Write
				\item[] \textbf{VIEW}
				\item[] View
				\item[] \textbf{AVER}
				\item[] Average
				\item[] \textbf{MAXH}
				\item[] Maximun Hold
				\item[] \textbf{MINH}
				\item[] Minumun Hold
				\item[] \textbf{RMS}
				\item[] RMS


			\end{itemize}\} \\
			Description: Get the type of display and evaluation of the traces. \\
			SCPI Command: DISPlay[:WINDow]:TRACe:MODE? \\
			\\ 

		\end{tabular}

		\begin{tabular}{N}
			\hline
			\bfseries TraceMode-Sel \\ \hline
			\emph{Trace Mode Selection} \\
			Data type: enum \\
			Description: Set the type of display and evaluation of the traces. \begin{itemize}[noitemsep]
				\small
				\item[] WRIT
				\item[] VIEW
				\item[] AVER
				\item[] MAXH
				\item[] MINH
				\item[] RMS
			\end{itemize} \\
			SCPI Command: DISPlay[:WINDow]:TRACe:MODE \emph{option} \\
			\\ 

		\end{tabular}





%%%%%%%%------------------------%%%%%%%%%%%%%%%



			Configuration needed:
				\begin{itemize}[noitemsep]
				\small
				\item[] SOUR:POW:MODE SWE
				\item[] SOUR:SWE:POW:MODE MAN
				\end{itemize} \\

		\end{tabular}










%%%%%%%%%%%%%%%%%%%%%%%%%%%%%%%%%%%%%%%%%%%%%%%%%%%%%%%%%%%%%%%%%%%%%%%%%%%%%%%%%%%%%%%%%%%%%%%%%%%%%%%%%%%%%%%%%%%%%%%%%%%%%%%%%%%%%%%%%%%%%%%%%%%%%%%%%%%%%%%

		\begin{tabular}{N}
			\hline
			\bfseries GeneralSearchOnce-Cmd \\ \hline
			\emph{Activate Level Control Correction Command} \\
			Data type: bool \{\begin{itemize}[noitemsep]
				\small
				\item[] OFF
				\item[] ON
			\end{itemize}\} \\
			Description: Temporarily activates level control for correction purposes. \\
			SCPI Command: SOUR:POW:ALC:SONC \emph{value} \\
			\\

		\end{tabular}


		\begin{tabular}{N}
			\hline
			\bfseries GeneralAttFixLow-Mon \\ \hline
			\emph{Minumum Level With Fixed Attenuator Monitor} \\
			Data type: float \\
			Unit: dB \\
			Description: Queries the minimum level which can be set when the attenuator is fixed. \\
			SCPI Command: OUTP:AFIX:RANG:LOW? \\
			Configuration needed: \begin{itemize}[noitemsep]
				\small
				\item[] OUTP:AMOD FIX
			\end{itemize} \\

		\end{tabular}




		\begin{tabular}{N}
			\hline
			\bfseries GeneralPwrMan-SP \\ \hline
			\emph{Sweep Step Level Set Point} \\
			Data type: float \\
			Unit: dBm \\
			Range: -145 dBm to 30 dBm \\
			Description: In Sweep mode, this PV sets the level for the next sweep step in the Step sweep mode. Here only level values between the settings of  Start Level (SOUR:POW:STAR) and Stop Level (SOUR:POW:STOP) are permitted. Each sweep step is triggered by a separate command. \\
			SCPI Command: SOUR:POW:MAN \emph{value} \\
			Configuration needed:
				\begin{itemize}[noitemsep]
				\small
				\item[] SOUR:POW:MODE SWE
				\item[] SOUR:SWE:POW:MODE MAN
				\end{itemize} \\

		\end{tabular}


		\begin{tabular}{N}
			\hline
			\bfseries GeneralPwrMan-RB \\ \hline
			\emph{Sweep Step Level Read Back} \\
			Data type: float \\
			Unit: dBm \\
			Description: Read the level for the next sweep step in the Step sweep mode. \\
			SCPI Command: SOUR:POW:MAN? \\
			\\

		\end{tabular}


		\begin{tabular}{N}
			\hline
			\bfseries GeneralAlc-Sel \\ \hline
			\emph{Automatic Level Control Selection} \\
			Data type: enum \\
			Description: Activates/deactivates automatic level control. \begin{itemize}[noitemsep]
				\small
				\item[] \textbf{ON}
				\item[] Internal level control is permanently activated.
				\item[] \textbf{OFF}
				\item[] Internal level control is deactivated; Sample and Hold mode is activated.
				\item[] \textbf{AUTO}
				\item[] Internal level control is activated/deactivated automatically, depending on the operatind state.
			\end{itemize} \\
			SCPI Command: SOUR:POW:ALC:STAT \emph{value} \\
			\\ 

		\end{tabular}


		\begin{tabular}{N}
			\hline
			\bfseries GeneralAlc-Sts \\ \hline
			\emph{Automatic Level Control Status} \\
			Data type: enum \{\begin{itemize}[noitemsep]
				\small
				\item[] ON
				\item[] OFF
				\item[] AUTO
			\end{itemize}\} \\
			Description: Shows the automatic level control state. \\
			SCPI Command: SOUR:POW:ALC:STAT? \\
			\\ 

		\end{tabular}


		\begin{tabular}{N}
			\hline
			\bfseries GeneralRFLvl-SP \\ \hline
			\emph{Level for the RF Signal Set Point} \\
			Data type: float \\
			Unit: dBm \\
			Range: -145 dBm to 30 dBm \\
			Description: Sets the RF level applied to the DUT (Device Under Test). If specified, a level offset is included. \\
			SCPI Command: SOUR:POW:LEV:IMM:AMPL \emph{value} \\
			\\
			
		\end{tabular}


		\begin{tabular}{N}
			\hline
			\bfseries GeneralRFLvl-RB \\ \hline
			\emph{Level for the RF Signal Read Back} \\
			Data type: float \\
			Unit: dBm \\
			Description: Reads the RF level applied to the DUT (Device Under Test). \\
			SCPI Command: SOUR:POW:LEV:IMM:AMPL? \\
			\\ 

		\end{tabular}


		\begin{tabular}{N}
			\hline
			\bfseries GeneralPwrLim-SP \\ \hline
			\emph{Limit of Maximum RF Output Level Set Point} \\
			Data type: float \\
			Unit: dBm \\
			Range: -145 dBm to 30 dBm \\
			Description: Limits the maximum RF output level in CW and SWEEP mode. It does not influence the "Level" display or the response to the POW? query command. \\
			SCPI Command: SOUR:POW:LIM:AMPL \emph{value} \\
			\\
			
		\end{tabular}


		\begin{tabular}{N}
			\hline
			\bfseries GeneralPwrLim-RB \\ \hline
			\emph{Limit of Maximum RF Output Level Read Back} \\
			Data type: float \\
			Unit: dBm \\
			Description: Reads the limit of maximum RF output level in CW and SWEEP mode. \\
			SCPI Command: SOUR:POW:LIM:AMPL? \\
			\\

		\end{tabular}


		\begin{tabular}{N}
			\hline
			\bfseries GeneralSearchOnce-Cmd \\ \hline
			\emph{Activate Level Control Correction Command} \\
			Data type: bool \{\begin{itemize}[noitemsep]
				\small
				\item[] OFF
				\item[] ON
			\end{itemize}\} \\
			Description: Temporarily activates level control for correction purposes. \\
			SCPI Command: SOUR:POW:ALC:SONC \emph{value} \\
			\\

		\end{tabular}


	 TABLE: Frequency Functionalities
	\subsection{Frequency Functionalities}\label{pvgroup:function} %LABEL NOT CHANGED YET

	\paragraph{}

		\begin{tabular}{N}
			\hline
			\bfseries FreqStartFreq-SP \\ \hline
			\emph{Start Frequency Set Point} \\
			Data type: float \\
			Unit: Hz \\
			Range: 9 kHz to 3 GHz \\
			Description: Sets the start frequency for the RF sweep. This parameter relates to the center frequency and span. If you change the frequency, these parameters change accordingly. \\
			SCPI Command: FREQ:STAR \emph{value} \\
			\\
			
		\end{tabular}


		\begin{tabular}{N}
			\hline
			\bfseries FreqStartFreq-RB \\ \hline
			\emph{Start Frequency Read Back} \\
			Data type: float \\
			Unit: Hz \\
			Description: Reads the start frequency for the RF sweep. \\
			SCPI Command: FREQ:STAR? \\
			\\

		\end{tabular}


		\begin{tabular}{N}
			\hline
			\bfseries FreqStopFreq-SP \\ \hline
			\emph{Stop Frequency Set Point} \\
			Data type: float \\
			Unit: Hz \\
			Range: 9 kHz to 3 GHz \\
			Description: Sets the stop frequency for the RF sweep.
This parameter is related to the center frequency and span. If you change the fre-
quency, these parameters change accordingly. \\
			SCPI Command: FREQ:STOP \emph{value} \\
			\\
			
		\end{tabular}


		\begin{tabular}{N}
			\hline
			\bfseries FreqStopFreq-RB \\ \hline
			\emph{Stop Frequency Read Back} \\
			Data type: float \\
			Unit: Hz \\
			Description: Reads the stop frequency for the RF sweep. \\
			SCPI Command: FREQ:STOP? \\
			\\

		\end{tabular}


		\begin{tabular}{N}
			\hline
			\bfseries FreqCenterFreq-SP \\ \hline
			\emph{Center Frequency Set Point} \\
			Data type: float \\
			Unit: Hz \\
			Range: 9 kHz to 3 GHz \\
			Description: Sets the center frequency of the RF sweep range. The range is defined by this center frequency and the specified Frequency Span (SOUR:FREQ:SPAN). \\
			SCPI Command: SOUR:FREQ:CENT \emph{value} \\
			\\
			
		\end{tabular}


		\begin{tabular}{N}
			\hline
			\bfseries FreqCenterFreq-RB \\ \hline
			\emph{Center Frequency Read Back} \\
			Data type: float \\
			Unit: Hz \\
			Description: Reads the center frequency for the RF frequency sweep. \\
			SCPI Command: SOUR:FREQ:CENT? \\
			\\

		\end{tabular}


		\begin{tabular}{N}
			\hline
			\bfseries FreqFreqSpan-SP \\ \hline
			\emph{Frequency Span Set Point} \\
			Data type: float \\
			Unit: Hz \\
			Range: 9 kHz to 3 GHz \\
			Description: Determines the extent of the frequency sweep range. This setting in combination with the Center Frequency setting (SOUR:FREQ:CENT) defines the sweep range. \\
			SCPI Command: SOUR:FREQ:SPAN \emph{value} \\
			\\
			
		\end{tabular}


		\begin{tabular}{N}
			\hline
			\bfseries FreqFreqSpan-RB \\ \hline
			\emph{Frequency Span Read Back} \\
			Data type: float \\
			Unit: Hz \\
			Description: Reads the extent of the frequency sweep range. \\
			SCPI Command: SOUR:FREQ:SPAN? \\
			\\

		\end{tabular}


		\begin{tabular}{N}
			\hline
			\bfseries FreqFSweepPts-SP \\ \hline
			\emph{Number of Frequency Sweep Steps Set Point} \\
			Data type: integer \\
			Range: 2 to Max
			Description: Determines the number of steps for the RF frequency sweep within the sweep range. This parameter always applies to the currently set sweep spacing (Linear or Logarithmic) and correlates with the step size. \\
			SCPI Command: SOUR:SWE:FREQ:POIN \emph{value} \\
			Configuration needed: \begin{itemize}[noitemsep]
				\small
				\item[] SOUR:SWE:FREQ:MODE MAN
			\end{itemize} \\
		
		\end{tabular}


		\begin{tabular}{N}
			\hline
			\bfseries FreqFSweepPts-RB \\ \hline
			\emph{Number of Frequency Sweep Steps Read Back} \\
			Data type: integer \\
			Description: Read number of steps for the RF frequency sweep. \\
			SCPI Command: SOUR:SWE:FREQ:POIN? \\
			\\

		\end{tabular}


		\begin{tabular}{N}
			\hline
			\bfseries FreqFreqRetr-Sel \\ \hline
			\emph{Retrace State for the Frequency Sweep Enable Selection} \\
			Data type: bool \{\begin{itemize}[noitemsep]
				\small
				\item[] OFF
				\item[] ON
			\end{itemize}\} \\
			Description: Selects if the signal changes to the start frequency value while it is waiting for the next trigger event. You can enable this feature, when you are working with sawtooth shapes in sweep mode "Single" or "External Single". \\
			SCPI Command: SOUR:SWE:FREQ:RETR \emph{value} \\
			Configuration needed: \begin{itemize}[noitemsep]
				\small
				\item[] SOUR:FREQ:MODE SWE
				\item[] SOUR:SWE:FREQ:MODE AUTO
				\item[] TRIG:FSW:SOUR MAN|EXT
			\end{itemize} \\

		\end{tabular}


		\begin{tabular}{N}
			\hline
			\bfseries FreqFreqRetr-Sts \\ \hline
			\emph{Retrace State for the Frequency Sweep Enable Status} \\
			Data type: bool \{\begin{itemize}[noitemsep]
				\small
				\item[] OFF
				\item[] ON
			\end{itemize}\} \\
			Description: Shows if the signal changes to the start frequency value while it is waiting for the next trigger event. \\
			SCPI Command: SOUR:SWE:FREQ:RETR? \\
			\\

		\end{tabular}


		\begin{tabular}{N}
			\hline
			\bfseries FreqFRunnMode-Mon \\ \hline
			\emph{Frequency Sweep Running Mode Monitor} \\
			Data type: bool \{\begin{itemize}[noitemsep]
				\small
				\item[] OFF
				\item[] ON
			\end{itemize}\} \\
			Description: Queries the current state of the frequency sweep mode. \\
			SCPI Command: SOUR:SWE:FREQ:RUNN? \\
			\\

		\end{tabular}


		\begin{tabular}{N}
			\hline
			\bfseries FreqFreqShp-Sel \\ \hline
			\emph{Sequence Shape for Level Sweep Selection} \\
			Data type: enum \\
			Description: Sets the wave shape for the frequency sweep.\begin{itemize}[noitemsep]
				\small
				\item[] \textbf{SAWTooth}
				\item[] The shape of the sweep sequence resembles a sawtooth. One sweep runs from start to stop frequency. Each subsequent sweep starts at the start frequency.
				\item[] \textbf{TRIangle}
				\item[] The shape of the sweep resembles a triangle. One sweep runs from start to stop frequency and back. Each subsequent sweep starts at the start frequency.
			\end{itemize} \\
			SCPI Command: SOUR:SWE:FREQ:MODE \emph{value} \\
			\\

		\end{tabular}


		\begin{tabular}{N}
			\hline
			\bfseries FreqFreqShp-Sts \\ \hline
			\emph{Sequence Shape Frequency Sweep Status} \\
			Data type: enum \{\begin{itemize}[noitemsep]
				\small
				\item[] SAWT
				\item[] TRI
			\end{itemize}\} \\
			Description: Shows the frequency sweep wave shape. \\
			SCPI Command: SOUR:SWE:FREQ:MODE? \\
			\\

		\end{tabular}


		\begin{tabular}{N}
			\hline
			\bfseries FreqFStepLog-SP \\ \hline
			\emph{Logarithmic Step for Frequency Sweep Set Point} \\
			Data type: float \\
			Unit: \% \\
			Range: 9 kHz to 3 GHz \\
			Description: Sets a logarithmically determined sweep step size for the RF frequency sweep. It is expressed in percent and you must enter the value and the unit PCT with the command. The frequency is increased by a logarithmically calculated fraction of the current frequency. \\
			SCPI Command: SOUR:SWE:FREQ:STEP:LOG \emph{value} \\
			\\
			
		\end{tabular}


		\begin{tabular}{N}
			\hline
			\bfseries FreqFStepLog-RB \\ \hline
			\emph{Logarithmic Step for Frequency Sweep Read Back} \\
			Data type: float \\
			Unit: \% \\
			Description: Reads a logarithmically determined sweep step size for the RF frequency sweep. \\
			SCPI Command: SOUR:SWE:FREQ:STEP:LOG? \\
			\\

		\end{tabular}


		\begin{tabular}{N}
			\hline
			\bfseries FreqPSweepPts-SP \\ \hline
			\emph{Number of Steps for Level Sweep Set Point} \\
			Data type: integer \\
			Unit: s \\
			Range: 2 to Max \\ 
			Description: Sets the number of steps for the RF level sweep within the sweep range. This parameter always applies to the currently set sweep spacing and correlates with the step size. If you change the number of sweep points, the step size changes accordingly. The sweep range remains the same. \\
			SCPI Command: SOUR:SWE:POW:POIN \emph{value} \\
			\\
			
		\end{tabular}


		\begin{tabular}{N}
			\hline
			\bfseries FreqPSweepPts-RB \\ \hline
			\emph{Number of Steps for Level Sweep Read Back} \\
			Data type: integer \\
			Unit: s \\
			Description: Reads the number of steps for the level sweep. \\
			SCPI Command: SOUR:SWE:POW:POIN? \\
			\\

		\end{tabular}



		\begin{tabular}{N}
			\hline
			\bfseries FreqLFStartFreq-SP \\ \hline
			\emph{Start Frequency for LF Sweep Set Point} \\
			Data type: float \\
			Unit: Hz \\ 
			Range: 9 kHz to 3 GHz
			Description: Sets the start frequency for the LF sweeep.\\
			SCPI Command: SOUR:LFO:FREQ:STAR \emph{value} \\
			\\
			
		\end{tabular}


		\begin{tabular}{N}
			\hline
			\bfseries FreqLFStartFreq-RB \\ \hline
			\emph{Start Frequency for LF Sweep Read Back} \\
			Data type: float \\
			Unit: Hz \\
			Description: Reads the start frequency for the LF sweep. \\
			SCPI Command: SOUR:LFO:FREQ:STAR? \\
			\\

		\end{tabular}


		\begin{tabular}{N}
			\hline
			\bfseries FreqLFStopFreq-SP \\ \hline
			\emph{Stop Frequency for LF Sweep Set Point} \\
			Data type: float \\
			Unit: Hz \\ 
			Range: 9 kHz to 3 GHz
			Description: Sets the stop frequency for the LF sweeep.\\
			SCPI Command: SOUR:LFO:FREQ:STOP \emph{value} \\
			\\
			
		\end{tabular}


		\begin{tabular}{N}
			\hline
			\bfseries FreqLFStopFreq-RB \\ \hline
			\emph{Stop Frequency for LF Sweep Read Back} \\
			Data type: float \\
			Unit: Hz \\
			Description: Reads the stop frequency for the LF sweep. \\
			SCPI Command: SOUR:LFO:FREQ:STOP? \\
			\\

		\end{tabular}



---------
\end{document}
\grid
